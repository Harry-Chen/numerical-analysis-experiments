
% Default to the notebook output style

    


% Inherit from the specified cell style.




    
\documentclass[11pt]{ctexart}

    
    
    \usepackage[T1]{fontenc}
    % Nicer default font (+ math font) than Computer Modern for most use cases
    \usepackage{mathpazo}

    % Basic figure setup, for now with no caption control since it's done
    % automatically by Pandoc (which extracts ![](path) syntax from Markdown).
    \usepackage{graphicx}
    % We will generate all images so they have a width \maxwidth. This means
    % that they will get their normal width if they fit onto the page, but
    % are scaled down if they would overflow the margins.
    \makeatletter
    \def\maxwidth{\ifdim\Gin@nat@width>\linewidth\linewidth
    \else\Gin@nat@width\fi}
    \makeatother
    \let\Oldincludegraphics\includegraphics
    % Set max figure width to be 80% of text width, for now hardcoded.
    \renewcommand{\includegraphics}[1]{\Oldincludegraphics[width=.8\maxwidth]{#1}}
    % Ensure that by default, figures have no caption (until we provide a
    % proper Figure object with a Caption API and a way to capture that
    % in the conversion process - todo).
    \usepackage{caption}
    \DeclareCaptionLabelFormat{nolabel}{}
    \captionsetup{labelformat=nolabel}

    \usepackage{adjustbox} % Used to constrain images to a maximum size 
    \usepackage{xcolor} % Allow colors to be defined
    \usepackage{enumerate} % Needed for markdown enumerations to work
    \usepackage{geometry} % Used to adjust the document margins
    \usepackage{amsmath} % Equations
    \usepackage{amssymb} % Equations
    \usepackage{textcomp} % defines textquotesingle
    % Hack from http://tex.stackexchange.com/a/47451/13684:
    \AtBeginDocument{%
        \def\PYZsq{\textquotesingle}% Upright quotes in Pygmentized code
    }
    \usepackage{upquote} % Upright quotes for verbatim code
    \usepackage{eurosym} % defines \euro
    \usepackage[mathletters]{ucs} % Extended unicode (utf-8) support
    \usepackage[utf8x]{inputenc} % Allow utf-8 characters in the tex document
    \usepackage{fancyvrb} % verbatim replacement that allows latex
    \usepackage{grffile} % extends the file name processing of package graphics 
                         % to support a larger range 
    % The hyperref package gives us a pdf with properly built
    % internal navigation ('pdf bookmarks' for the table of contents,
    % internal cross-reference links, web links for URLs, etc.)
    \usepackage{hyperref}
    \usepackage{longtable} % longtable support required by pandoc >1.10
    \usepackage{booktabs}  % table support for pandoc > 1.12.2
    \usepackage[inline]{enumitem} % IRkernel/repr support (it uses the enumerate* environment)
    \usepackage[normalem]{ulem} % ulem is needed to support strikethroughs (\sout)
                                % normalem makes italics be italics, not underlines
    

    
    
    % Colors for the hyperref package
    \definecolor{urlcolor}{rgb}{0,.145,.698}
    \definecolor{linkcolor}{rgb}{.71,0.21,0.01}
    \definecolor{citecolor}{rgb}{.12,.54,.11}

    % ANSI colors
    \definecolor{ansi-black}{HTML}{3E424D}
    \definecolor{ansi-black-intense}{HTML}{282C36}
    \definecolor{ansi-red}{HTML}{E75C58}
    \definecolor{ansi-red-intense}{HTML}{B22B31}
    \definecolor{ansi-green}{HTML}{00A250}
    \definecolor{ansi-green-intense}{HTML}{007427}
    \definecolor{ansi-yellow}{HTML}{DDB62B}
    \definecolor{ansi-yellow-intense}{HTML}{B27D12}
    \definecolor{ansi-blue}{HTML}{208FFB}
    \definecolor{ansi-blue-intense}{HTML}{0065CA}
    \definecolor{ansi-magenta}{HTML}{D160C4}
    \definecolor{ansi-magenta-intense}{HTML}{A03196}
    \definecolor{ansi-cyan}{HTML}{60C6C8}
    \definecolor{ansi-cyan-intense}{HTML}{258F8F}
    \definecolor{ansi-white}{HTML}{C5C1B4}
    \definecolor{ansi-white-intense}{HTML}{A1A6B2}

    % commands and environments needed by pandoc snippets
    % extracted from the output of `pandoc -s`
    \providecommand{\tightlist}{%
      \setlength{\itemsep}{0pt}\setlength{\parskip}{0pt}}
    \DefineVerbatimEnvironment{Highlighting}{Verbatim}{commandchars=\\\{\}}
    % Add ',fontsize=\small' for more characters per line
    \newenvironment{Shaded}{}{}
    \newcommand{\KeywordTok}[1]{\textcolor[rgb]{0.00,0.44,0.13}{\textbf{{#1}}}}
    \newcommand{\DataTypeTok}[1]{\textcolor[rgb]{0.56,0.13,0.00}{{#1}}}
    \newcommand{\DecValTok}[1]{\textcolor[rgb]{0.25,0.63,0.44}{{#1}}}
    \newcommand{\BaseNTok}[1]{\textcolor[rgb]{0.25,0.63,0.44}{{#1}}}
    \newcommand{\FloatTok}[1]{\textcolor[rgb]{0.25,0.63,0.44}{{#1}}}
    \newcommand{\CharTok}[1]{\textcolor[rgb]{0.25,0.44,0.63}{{#1}}}
    \newcommand{\StringTok}[1]{\textcolor[rgb]{0.25,0.44,0.63}{{#1}}}
    \newcommand{\CommentTok}[1]{\textcolor[rgb]{0.38,0.63,0.69}{\textit{{#1}}}}
    \newcommand{\OtherTok}[1]{\textcolor[rgb]{0.00,0.44,0.13}{{#1}}}
    \newcommand{\AlertTok}[1]{\textcolor[rgb]{1.00,0.00,0.00}{\textbf{{#1}}}}
    \newcommand{\FunctionTok}[1]{\textcolor[rgb]{0.02,0.16,0.49}{{#1}}}
    \newcommand{\RegionMarkerTok}[1]{{#1}}
    \newcommand{\ErrorTok}[1]{\textcolor[rgb]{1.00,0.00,0.00}{\textbf{{#1}}}}
    \newcommand{\NormalTok}[1]{{#1}}
    
    % Additional commands for more recent versions of Pandoc
    \newcommand{\ConstantTok}[1]{\textcolor[rgb]{0.53,0.00,0.00}{{#1}}}
    \newcommand{\SpecialCharTok}[1]{\textcolor[rgb]{0.25,0.44,0.63}{{#1}}}
    \newcommand{\VerbatimStringTok}[1]{\textcolor[rgb]{0.25,0.44,0.63}{{#1}}}
    \newcommand{\SpecialStringTok}[1]{\textcolor[rgb]{0.73,0.40,0.53}{{#1}}}
    \newcommand{\ImportTok}[1]{{#1}}
    \newcommand{\DocumentationTok}[1]{\textcolor[rgb]{0.73,0.13,0.13}{\textit{{#1}}}}
    \newcommand{\AnnotationTok}[1]{\textcolor[rgb]{0.38,0.63,0.69}{\textbf{\textit{{#1}}}}}
    \newcommand{\CommentVarTok}[1]{\textcolor[rgb]{0.38,0.63,0.69}{\textbf{\textit{{#1}}}}}
    \newcommand{\VariableTok}[1]{\textcolor[rgb]{0.10,0.09,0.49}{{#1}}}
    \newcommand{\ControlFlowTok}[1]{\textcolor[rgb]{0.00,0.44,0.13}{\textbf{{#1}}}}
    \newcommand{\OperatorTok}[1]{\textcolor[rgb]{0.40,0.40,0.40}{{#1}}}
    \newcommand{\BuiltInTok}[1]{{#1}}
    \newcommand{\ExtensionTok}[1]{{#1}}
    \newcommand{\PreprocessorTok}[1]{\textcolor[rgb]{0.74,0.48,0.00}{{#1}}}
    \newcommand{\AttributeTok}[1]{\textcolor[rgb]{0.49,0.56,0.16}{{#1}}}
    \newcommand{\InformationTok}[1]{\textcolor[rgb]{0.38,0.63,0.69}{\textbf{\textit{{#1}}}}}
    \newcommand{\WarningTok}[1]{\textcolor[rgb]{0.38,0.63,0.69}{\textbf{\textit{{#1}}}}}
    
    
    % Define a nice break command that doesn't care if a line doesn't already
    % exist.
    \def\br{\hspace*{\fill} \\* }
    % Math Jax compatability definitions
    \def\gt{>}
    \def\lt{<}
    % Document parameters
    \title{Chapter5}
    
    
    

    % Pygments definitions
    
\makeatletter
\def\PY@reset{\let\PY@it=\relax \let\PY@bf=\relax%
    \let\PY@ul=\relax \let\PY@tc=\relax%
    \let\PY@bc=\relax \let\PY@ff=\relax}
\def\PY@tok#1{\csname PY@tok@#1\endcsname}
\def\PY@toks#1+{\ifx\relax#1\empty\else%
    \PY@tok{#1}\expandafter\PY@toks\fi}
\def\PY@do#1{\PY@bc{\PY@tc{\PY@ul{%
    \PY@it{\PY@bf{\PY@ff{#1}}}}}}}
\def\PY#1#2{\PY@reset\PY@toks#1+\relax+\PY@do{#2}}

\expandafter\def\csname PY@tok@w\endcsname{\def\PY@tc##1{\textcolor[rgb]{0.73,0.73,0.73}{##1}}}
\expandafter\def\csname PY@tok@c\endcsname{\let\PY@it=\textit\def\PY@tc##1{\textcolor[rgb]{0.25,0.50,0.50}{##1}}}
\expandafter\def\csname PY@tok@cp\endcsname{\def\PY@tc##1{\textcolor[rgb]{0.74,0.48,0.00}{##1}}}
\expandafter\def\csname PY@tok@k\endcsname{\let\PY@bf=\textbf\def\PY@tc##1{\textcolor[rgb]{0.00,0.50,0.00}{##1}}}
\expandafter\def\csname PY@tok@kp\endcsname{\def\PY@tc##1{\textcolor[rgb]{0.00,0.50,0.00}{##1}}}
\expandafter\def\csname PY@tok@kt\endcsname{\def\PY@tc##1{\textcolor[rgb]{0.69,0.00,0.25}{##1}}}
\expandafter\def\csname PY@tok@o\endcsname{\def\PY@tc##1{\textcolor[rgb]{0.40,0.40,0.40}{##1}}}
\expandafter\def\csname PY@tok@ow\endcsname{\let\PY@bf=\textbf\def\PY@tc##1{\textcolor[rgb]{0.67,0.13,1.00}{##1}}}
\expandafter\def\csname PY@tok@nb\endcsname{\def\PY@tc##1{\textcolor[rgb]{0.00,0.50,0.00}{##1}}}
\expandafter\def\csname PY@tok@nf\endcsname{\def\PY@tc##1{\textcolor[rgb]{0.00,0.00,1.00}{##1}}}
\expandafter\def\csname PY@tok@nc\endcsname{\let\PY@bf=\textbf\def\PY@tc##1{\textcolor[rgb]{0.00,0.00,1.00}{##1}}}
\expandafter\def\csname PY@tok@nn\endcsname{\let\PY@bf=\textbf\def\PY@tc##1{\textcolor[rgb]{0.00,0.00,1.00}{##1}}}
\expandafter\def\csname PY@tok@ne\endcsname{\let\PY@bf=\textbf\def\PY@tc##1{\textcolor[rgb]{0.82,0.25,0.23}{##1}}}
\expandafter\def\csname PY@tok@nv\endcsname{\def\PY@tc##1{\textcolor[rgb]{0.10,0.09,0.49}{##1}}}
\expandafter\def\csname PY@tok@no\endcsname{\def\PY@tc##1{\textcolor[rgb]{0.53,0.00,0.00}{##1}}}
\expandafter\def\csname PY@tok@nl\endcsname{\def\PY@tc##1{\textcolor[rgb]{0.63,0.63,0.00}{##1}}}
\expandafter\def\csname PY@tok@ni\endcsname{\let\PY@bf=\textbf\def\PY@tc##1{\textcolor[rgb]{0.60,0.60,0.60}{##1}}}
\expandafter\def\csname PY@tok@na\endcsname{\def\PY@tc##1{\textcolor[rgb]{0.49,0.56,0.16}{##1}}}
\expandafter\def\csname PY@tok@nt\endcsname{\let\PY@bf=\textbf\def\PY@tc##1{\textcolor[rgb]{0.00,0.50,0.00}{##1}}}
\expandafter\def\csname PY@tok@nd\endcsname{\def\PY@tc##1{\textcolor[rgb]{0.67,0.13,1.00}{##1}}}
\expandafter\def\csname PY@tok@s\endcsname{\def\PY@tc##1{\textcolor[rgb]{0.73,0.13,0.13}{##1}}}
\expandafter\def\csname PY@tok@sd\endcsname{\let\PY@it=\textit\def\PY@tc##1{\textcolor[rgb]{0.73,0.13,0.13}{##1}}}
\expandafter\def\csname PY@tok@si\endcsname{\let\PY@bf=\textbf\def\PY@tc##1{\textcolor[rgb]{0.73,0.40,0.53}{##1}}}
\expandafter\def\csname PY@tok@se\endcsname{\let\PY@bf=\textbf\def\PY@tc##1{\textcolor[rgb]{0.73,0.40,0.13}{##1}}}
\expandafter\def\csname PY@tok@sr\endcsname{\def\PY@tc##1{\textcolor[rgb]{0.73,0.40,0.53}{##1}}}
\expandafter\def\csname PY@tok@ss\endcsname{\def\PY@tc##1{\textcolor[rgb]{0.10,0.09,0.49}{##1}}}
\expandafter\def\csname PY@tok@sx\endcsname{\def\PY@tc##1{\textcolor[rgb]{0.00,0.50,0.00}{##1}}}
\expandafter\def\csname PY@tok@m\endcsname{\def\PY@tc##1{\textcolor[rgb]{0.40,0.40,0.40}{##1}}}
\expandafter\def\csname PY@tok@gh\endcsname{\let\PY@bf=\textbf\def\PY@tc##1{\textcolor[rgb]{0.00,0.00,0.50}{##1}}}
\expandafter\def\csname PY@tok@gu\endcsname{\let\PY@bf=\textbf\def\PY@tc##1{\textcolor[rgb]{0.50,0.00,0.50}{##1}}}
\expandafter\def\csname PY@tok@gd\endcsname{\def\PY@tc##1{\textcolor[rgb]{0.63,0.00,0.00}{##1}}}
\expandafter\def\csname PY@tok@gi\endcsname{\def\PY@tc##1{\textcolor[rgb]{0.00,0.63,0.00}{##1}}}
\expandafter\def\csname PY@tok@gr\endcsname{\def\PY@tc##1{\textcolor[rgb]{1.00,0.00,0.00}{##1}}}
\expandafter\def\csname PY@tok@ge\endcsname{\let\PY@it=\textit}
\expandafter\def\csname PY@tok@gs\endcsname{\let\PY@bf=\textbf}
\expandafter\def\csname PY@tok@gp\endcsname{\let\PY@bf=\textbf\def\PY@tc##1{\textcolor[rgb]{0.00,0.00,0.50}{##1}}}
\expandafter\def\csname PY@tok@go\endcsname{\def\PY@tc##1{\textcolor[rgb]{0.53,0.53,0.53}{##1}}}
\expandafter\def\csname PY@tok@gt\endcsname{\def\PY@tc##1{\textcolor[rgb]{0.00,0.27,0.87}{##1}}}
\expandafter\def\csname PY@tok@err\endcsname{\def\PY@bc##1{\setlength{\fboxsep}{0pt}\fcolorbox[rgb]{1.00,0.00,0.00}{1,1,1}{\strut ##1}}}
\expandafter\def\csname PY@tok@kc\endcsname{\let\PY@bf=\textbf\def\PY@tc##1{\textcolor[rgb]{0.00,0.50,0.00}{##1}}}
\expandafter\def\csname PY@tok@kd\endcsname{\let\PY@bf=\textbf\def\PY@tc##1{\textcolor[rgb]{0.00,0.50,0.00}{##1}}}
\expandafter\def\csname PY@tok@kn\endcsname{\let\PY@bf=\textbf\def\PY@tc##1{\textcolor[rgb]{0.00,0.50,0.00}{##1}}}
\expandafter\def\csname PY@tok@kr\endcsname{\let\PY@bf=\textbf\def\PY@tc##1{\textcolor[rgb]{0.00,0.50,0.00}{##1}}}
\expandafter\def\csname PY@tok@bp\endcsname{\def\PY@tc##1{\textcolor[rgb]{0.00,0.50,0.00}{##1}}}
\expandafter\def\csname PY@tok@fm\endcsname{\def\PY@tc##1{\textcolor[rgb]{0.00,0.00,1.00}{##1}}}
\expandafter\def\csname PY@tok@vc\endcsname{\def\PY@tc##1{\textcolor[rgb]{0.10,0.09,0.49}{##1}}}
\expandafter\def\csname PY@tok@vg\endcsname{\def\PY@tc##1{\textcolor[rgb]{0.10,0.09,0.49}{##1}}}
\expandafter\def\csname PY@tok@vi\endcsname{\def\PY@tc##1{\textcolor[rgb]{0.10,0.09,0.49}{##1}}}
\expandafter\def\csname PY@tok@vm\endcsname{\def\PY@tc##1{\textcolor[rgb]{0.10,0.09,0.49}{##1}}}
\expandafter\def\csname PY@tok@sa\endcsname{\def\PY@tc##1{\textcolor[rgb]{0.73,0.13,0.13}{##1}}}
\expandafter\def\csname PY@tok@sb\endcsname{\def\PY@tc##1{\textcolor[rgb]{0.73,0.13,0.13}{##1}}}
\expandafter\def\csname PY@tok@sc\endcsname{\def\PY@tc##1{\textcolor[rgb]{0.73,0.13,0.13}{##1}}}
\expandafter\def\csname PY@tok@dl\endcsname{\def\PY@tc##1{\textcolor[rgb]{0.73,0.13,0.13}{##1}}}
\expandafter\def\csname PY@tok@s2\endcsname{\def\PY@tc##1{\textcolor[rgb]{0.73,0.13,0.13}{##1}}}
\expandafter\def\csname PY@tok@sh\endcsname{\def\PY@tc##1{\textcolor[rgb]{0.73,0.13,0.13}{##1}}}
\expandafter\def\csname PY@tok@s1\endcsname{\def\PY@tc##1{\textcolor[rgb]{0.73,0.13,0.13}{##1}}}
\expandafter\def\csname PY@tok@mb\endcsname{\def\PY@tc##1{\textcolor[rgb]{0.40,0.40,0.40}{##1}}}
\expandafter\def\csname PY@tok@mf\endcsname{\def\PY@tc##1{\textcolor[rgb]{0.40,0.40,0.40}{##1}}}
\expandafter\def\csname PY@tok@mh\endcsname{\def\PY@tc##1{\textcolor[rgb]{0.40,0.40,0.40}{##1}}}
\expandafter\def\csname PY@tok@mi\endcsname{\def\PY@tc##1{\textcolor[rgb]{0.40,0.40,0.40}{##1}}}
\expandafter\def\csname PY@tok@il\endcsname{\def\PY@tc##1{\textcolor[rgb]{0.40,0.40,0.40}{##1}}}
\expandafter\def\csname PY@tok@mo\endcsname{\def\PY@tc##1{\textcolor[rgb]{0.40,0.40,0.40}{##1}}}
\expandafter\def\csname PY@tok@ch\endcsname{\let\PY@it=\textit\def\PY@tc##1{\textcolor[rgb]{0.25,0.50,0.50}{##1}}}
\expandafter\def\csname PY@tok@cm\endcsname{\let\PY@it=\textit\def\PY@tc##1{\textcolor[rgb]{0.25,0.50,0.50}{##1}}}
\expandafter\def\csname PY@tok@cpf\endcsname{\let\PY@it=\textit\def\PY@tc##1{\textcolor[rgb]{0.25,0.50,0.50}{##1}}}
\expandafter\def\csname PY@tok@c1\endcsname{\let\PY@it=\textit\def\PY@tc##1{\textcolor[rgb]{0.25,0.50,0.50}{##1}}}
\expandafter\def\csname PY@tok@cs\endcsname{\let\PY@it=\textit\def\PY@tc##1{\textcolor[rgb]{0.25,0.50,0.50}{##1}}}

\def\PYZbs{\char`\\}
\def\PYZus{\char`\_}
\def\PYZob{\char`\{}
\def\PYZcb{\char`\}}
\def\PYZca{\char`\^}
\def\PYZam{\char`\&}
\def\PYZlt{\char`\<}
\def\PYZgt{\char`\>}
\def\PYZsh{\char`\#}
\def\PYZpc{\char`\%}
\def\PYZdl{\char`\$}
\def\PYZhy{\char`\-}
\def\PYZsq{\char`\'}
\def\PYZdq{\char`\"}
\def\PYZti{\char`\~}
% for compatibility with earlier versions
\def\PYZat{@}
\def\PYZlb{[}
\def\PYZrb{]}
\makeatother


    % Exact colors from NB
    \definecolor{incolor}{rgb}{0.0, 0.0, 0.5}
    \definecolor{outcolor}{rgb}{0.545, 0.0, 0.0}



    
    % Prevent overflowing lines due to hard-to-break entities
    \sloppy 
    % Setup hyperref package
    \hypersetup{
      breaklinks=true,  % so long urls are correctly broken across lines
      colorlinks=true,
      urlcolor=urlcolor,
      linkcolor=linkcolor,
      citecolor=citecolor,
      }
    % Slightly bigger margins than the latex defaults
    
    \geometry{verbose,tmargin=1in,bmargin=1in,lmargin=1in,rmargin=1in}
    
    

    \begin{document}
    
	\title{数值分析实验五}
	\author{计63\,\,陈晟祺\,\,2016010981}
    
    \maketitle
    
    
    \subsection{上机题1}\label{ux4e0aux673aux98981}

\subsubsection{实验概述}\label{ux5b9eux9a8cux6982ux8ff0}

本实验要求用幂法求矩阵模最大的特征值 \(\lambda_1\) 和其对应的特征向量
\(\mathbf{x}_1\),并控制迭代前后误差小于 \(10^{-5}\)。

\subsubsection{实验过程}\label{ux5b9eux9a8cux8fc7ux7a0b}

幂法的实现比较简单,只需按照算法5.1描述的规则进行迭代即可。

    \begin{Verbatim}[commandchars=\\\{\}]
{\color{incolor}In [{\color{incolor}1}]:} \PY{k+kn}{import} \PY{n+nn}{numpy} \PY{k}{as} \PY{n+nn}{np}
        
        \PY{k}{def} \PY{n+nf}{power\PYZus{}method}\PY{p}{(}\PY{n}{A}\PY{p}{)}\PY{p}{:}
            \PY{n}{n} \PY{o}{=} \PY{n}{A}\PY{o}{.}\PY{n}{shape}\PY{p}{[}\PY{l+m+mi}{0}\PY{p}{]}
            \PY{n}{u} \PY{o}{=} \PY{n}{np}\PY{o}{.}\PY{n}{random}\PY{o}{.}\PY{n}{normal}\PY{p}{(}\PY{l+m+mf}{0.0}\PY{p}{,} \PY{l+m+mf}{1.0}\PY{p}{,} \PY{p}{(}\PY{n}{n}\PY{p}{,}\PY{l+m+mi}{1}\PY{p}{)}\PY{p}{)}
            \PY{n}{l} \PY{o}{=} \PY{l+m+mi}{0}
            \PY{c+c1}{\PYZsh{} iteration for lambda}
            \PY{k}{while} \PY{k+kc}{True}\PY{p}{:}
                \PY{n}{v} \PY{o}{=} \PY{n}{np}\PY{o}{.}\PY{n}{dot}\PY{p}{(}\PY{n}{A}\PY{p}{,} \PY{n}{u}\PY{p}{)}
                \PY{n}{new\PYZus{}l} \PY{o}{=} \PY{n}{v}\PY{p}{[}\PY{n}{np}\PY{o}{.}\PY{n}{argmax}\PY{p}{(}\PY{n}{np}\PY{o}{.}\PY{n}{abs}\PY{p}{(}\PY{n}{v}\PY{p}{)}\PY{p}{)}\PY{p}{]} \PY{c+c1}{\PYZsh{} approximation of lambda\PYZus{}1}
                \PY{n}{u} \PY{o}{=} \PY{n}{v} \PY{o}{/} \PY{n}{new\PYZus{}l}
                \PY{k}{if} \PY{n}{np}\PY{o}{.}\PY{n}{abs}\PY{p}{(}\PY{n}{new\PYZus{}l} \PY{o}{\PYZhy{}} \PY{n}{l}\PY{p}{)} \PY{o}{\PYZlt{}} \PY{l+m+mf}{1e\PYZhy{}5}\PY{p}{:}
                    \PY{k}{return} \PY{n}{new\PYZus{}l}\PY{p}{[}\PY{l+m+mi}{0}\PY{p}{]}\PY{p}{,} \PY{n}{u}
                \PY{n}{l} \PY{o}{=} \PY{n}{new\PYZus{}l}
\end{Verbatim}


    使用幂法分别对所给的两个矩阵进行迭代:

    \begin{Verbatim}[commandchars=\\\{\}]
{\color{incolor}In [{\color{incolor}2}]:} \PY{n}{A} \PY{o}{=} \PY{n}{np}\PY{o}{.}\PY{n}{array}\PY{p}{(}\PY{p}{[}\PY{p}{[}\PY{l+m+mi}{5}\PY{p}{,} \PY{o}{\PYZhy{}}\PY{l+m+mi}{4}\PY{p}{,} \PY{l+m+mi}{1}\PY{p}{]}\PY{p}{,} \PY{p}{[}\PY{o}{\PYZhy{}}\PY{l+m+mi}{4}\PY{p}{,} \PY{l+m+mi}{6}\PY{p}{,} \PY{o}{\PYZhy{}}\PY{l+m+mi}{4}\PY{p}{]}\PY{p}{,} \PY{p}{[}\PY{l+m+mi}{1}\PY{p}{,} \PY{o}{\PYZhy{}}\PY{l+m+mi}{4}\PY{p}{,} \PY{l+m+mi}{7}\PY{p}{]}\PY{p}{]}\PY{p}{)}
        \PY{n}{B} \PY{o}{=} \PY{n}{np}\PY{o}{.}\PY{n}{array}\PY{p}{(}\PY{p}{[}\PY{p}{[}\PY{l+m+mi}{25}\PY{p}{,} \PY{o}{\PYZhy{}}\PY{l+m+mi}{41}\PY{p}{,} \PY{l+m+mi}{10}\PY{p}{,} \PY{o}{\PYZhy{}}\PY{l+m+mi}{6}\PY{p}{]}\PY{p}{,} \PY{p}{[}\PY{o}{\PYZhy{}}\PY{l+m+mi}{41}\PY{p}{,} \PY{l+m+mi}{68}\PY{p}{,} \PY{o}{\PYZhy{}}\PY{l+m+mi}{17}\PY{p}{,} \PY{l+m+mi}{10}\PY{p}{]}\PY{p}{,} \PY{p}{[}\PY{l+m+mi}{10}\PY{p}{,} \PY{o}{\PYZhy{}}\PY{l+m+mi}{17}\PY{p}{,} \PY{l+m+mi}{5}\PY{p}{,} \PY{o}{\PYZhy{}}\PY{l+m+mi}{3}\PY{p}{]}\PY{p}{,} \PY{p}{[}\PY{o}{\PYZhy{}}\PY{l+m+mi}{6}\PY{p}{,} \PY{l+m+mi}{10}\PY{p}{,} \PY{o}{\PYZhy{}}\PY{l+m+mi}{3}\PY{p}{,} \PY{l+m+mi}{2}\PY{p}{]}\PY{p}{]}\PY{p}{)}
        
        \PY{n}{l\PYZus{}a}\PY{p}{,} \PY{n}{x\PYZus{}a} \PY{o}{=} \PY{n}{power\PYZus{}method}\PY{p}{(}\PY{n}{A}\PY{p}{)}
        \PY{n}{l\PYZus{}b}\PY{p}{,} \PY{n}{x\PYZus{}b} \PY{o}{=} \PY{n}{power\PYZus{}method}\PY{p}{(}\PY{n}{B}\PY{p}{)}
        
        \PY{n+nb}{print}\PY{p}{(}\PY{l+s+s1}{\PYZsq{}}\PY{l+s+s1}{A has main eigenvalue }\PY{l+s+si}{\PYZob{}:.5f\PYZcb{}}\PY{l+s+s1}{ with eigenvector }\PY{l+s+si}{\PYZob{}\PYZcb{}}\PY{l+s+s1}{\PYZsq{}}\PY{o}{.}\PY{n}{format}\PY{p}{(}\PY{n}{l\PYZus{}a}\PY{p}{,} \PY{n+nb}{list}\PY{p}{(}\PY{n+nb}{map}\PY{p}{(}\PY{l+s+s1}{\PYZsq{}}\PY{l+s+si}{\PYZob{}:.5f\PYZcb{}}\PY{l+s+s1}{\PYZsq{}}\PY{o}{.}\PY{n}{format}\PY{p}{,}\PY{n}{x\PYZus{}a}\PY{o}{.}\PY{n}{flatten}\PY{p}{(}\PY{p}{)}\PY{p}{)}\PY{p}{)}\PY{p}{)}\PY{p}{)}
        \PY{n+nb}{print}\PY{p}{(}\PY{l+s+s1}{\PYZsq{}}\PY{l+s+s1}{B has main eigenvalue }\PY{l+s+si}{\PYZob{}:.5f\PYZcb{}}\PY{l+s+s1}{ with eigenvector }\PY{l+s+si}{\PYZob{}\PYZcb{}}\PY{l+s+s1}{\PYZsq{}}\PY{o}{.}\PY{n}{format}\PY{p}{(}\PY{n}{l\PYZus{}b}\PY{p}{,} \PY{n+nb}{list}\PY{p}{(}\PY{n+nb}{map}\PY{p}{(}\PY{l+s+s1}{\PYZsq{}}\PY{l+s+si}{\PYZob{}:.5f\PYZcb{}}\PY{l+s+s1}{\PYZsq{}}\PY{o}{.}\PY{n}{format}\PY{p}{,}\PY{n}{x\PYZus{}b}\PY{o}{.}\PY{n}{flatten}\PY{p}{(}\PY{p}{)}\PY{p}{)}\PY{p}{)}\PY{p}{)}\PY{p}{)}
\end{Verbatim}


    \begin{Verbatim}[commandchars=\\\{\}]
A has main eigenvalue 12.25432 with eigenvector ['-0.67402', '1.00000', '-0.88956']
B has main eigenvalue 98.52170 with eigenvector ['-0.60397', '1.00000', '-0.25114', '0.14895']

    \end{Verbatim}

    使用 numpy 内置函数求值进行比较:

    \begin{Verbatim}[commandchars=\\\{\}]
{\color{incolor}In [{\color{incolor}3}]:} \PY{k}{def} \PY{n+nf}{np\PYZus{}method}\PY{p}{(}\PY{n}{A}\PY{p}{,} \PY{n}{ref\PYZus{}eig\PYZus{}vec}\PY{p}{)}\PY{p}{:}
            \PY{n}{w\PYZus{}a}\PY{p}{,} \PY{n}{v\PYZus{}a} \PY{o}{=} \PY{n}{np}\PY{o}{.}\PY{n}{linalg}\PY{o}{.}\PY{n}{eig}\PY{p}{(}\PY{n}{A}\PY{p}{)}
            \PY{n}{a\PYZus{}main\PYZus{}pos} \PY{o}{=} \PY{n}{np}\PY{o}{.}\PY{n}{argmax}\PY{p}{(}\PY{n}{np}\PY{o}{.}\PY{n}{abs}\PY{p}{(}\PY{n}{w\PYZus{}a}\PY{p}{)}\PY{p}{)} \PY{c+c1}{\PYZsh{} find main eigenvalue}
            \PY{n}{l\PYZus{}a}\PY{p}{,} \PY{n}{x\PYZus{}a} \PY{o}{=} \PY{n}{w\PYZus{}a}\PY{p}{[}\PY{n}{a\PYZus{}main\PYZus{}pos}\PY{p}{]}\PY{p}{,} \PY{n}{v\PYZus{}a}\PY{p}{[}\PY{p}{:}\PY{p}{,}\PY{n}{a\PYZus{}main\PYZus{}pos}\PY{p}{]}
            \PY{k}{return} \PY{n}{l\PYZus{}a}\PY{p}{,} \PY{n}{x\PYZus{}a} \PY{o}{/} \PY{n}{x\PYZus{}a}\PY{p}{[}\PY{n}{np}\PY{o}{.}\PY{n}{where}\PY{p}{(}\PY{n}{ref\PYZus{}eig\PYZus{}vec} \PY{o}{==} \PY{l+m+mf}{1.}\PY{p}{)}\PY{p}{[}\PY{l+m+mi}{0}\PY{p}{]}\PY{p}{]} \PY{c+c1}{\PYZsh{} do the same normalize as power method do}
        
        \PY{n}{l\PYZus{}a\PYZus{}n}\PY{p}{,} \PY{n}{x\PYZus{}a\PYZus{}n} \PY{o}{=} \PY{n}{np\PYZus{}method}\PY{p}{(}\PY{n}{A}\PY{p}{,} \PY{n}{x\PYZus{}a}\PY{p}{)}
        \PY{n}{l\PYZus{}b\PYZus{}n}\PY{p}{,} \PY{n}{x\PYZus{}b\PYZus{}n} \PY{o}{=} \PY{n}{np\PYZus{}method}\PY{p}{(}\PY{n}{B}\PY{p}{,} \PY{n}{x\PYZus{}b}\PY{p}{)}
        
        \PY{n+nb}{print}\PY{p}{(}\PY{l+s+s1}{\PYZsq{}}\PY{l+s+s1}{A has main eigenvalue }\PY{l+s+si}{\PYZob{}:.5f\PYZcb{}}\PY{l+s+s1}{ with eigenvector }\PY{l+s+si}{\PYZob{}\PYZcb{}}\PY{l+s+s1}{\PYZsq{}}\PY{o}{.}\PY{n}{format}\PY{p}{(}\PY{n}{l\PYZus{}a\PYZus{}n}\PY{p}{,} \PY{n+nb}{list}\PY{p}{(}\PY{n+nb}{map}\PY{p}{(}\PY{l+s+s1}{\PYZsq{}}\PY{l+s+si}{\PYZob{}:.5f\PYZcb{}}\PY{l+s+s1}{\PYZsq{}}\PY{o}{.}\PY{n}{format}\PY{p}{,}\PY{n}{x\PYZus{}a\PYZus{}n}\PY{o}{.}\PY{n}{flatten}\PY{p}{(}\PY{p}{)}\PY{p}{)}\PY{p}{)}\PY{p}{)}\PY{p}{)}
        \PY{n+nb}{print}\PY{p}{(}\PY{l+s+s1}{\PYZsq{}}\PY{l+s+s1}{B has main eigenvalue }\PY{l+s+si}{\PYZob{}:.5f\PYZcb{}}\PY{l+s+s1}{ with eigenvector }\PY{l+s+si}{\PYZob{}\PYZcb{}}\PY{l+s+s1}{\PYZsq{}}\PY{o}{.}\PY{n}{format}\PY{p}{(}\PY{n}{l\PYZus{}b\PYZus{}n}\PY{p}{,} \PY{n+nb}{list}\PY{p}{(}\PY{n+nb}{map}\PY{p}{(}\PY{l+s+s1}{\PYZsq{}}\PY{l+s+si}{\PYZob{}:.5f\PYZcb{}}\PY{l+s+s1}{\PYZsq{}}\PY{o}{.}\PY{n}{format}\PY{p}{,}\PY{n}{x\PYZus{}b\PYZus{}n}\PY{o}{.}\PY{n}{flatten}\PY{p}{(}\PY{p}{)}\PY{p}{)}\PY{p}{)}\PY{p}{)}\PY{p}{)}
\end{Verbatim}


    \begin{Verbatim}[commandchars=\\\{\}]
A has main eigenvalue 12.25432 with eigenvector ['-0.67402', '1.00000', '-0.88956']
B has main eigenvalue 98.52170 with eigenvector ['-0.60397', '1.00000', '-0.25114', '0.14895']

    \end{Verbatim}

    求得的主特征值小数点后五位都是一致的,并且当采用同样的归一化系数时,特征向量也是相同的。可见幂法的实现是正确的。

    \subsection{上机题 3}\label{ux4e0aux673aux9898-3}

\subsubsection{实验概述}\label{ux5b9eux9a8cux6982ux8ff0}

本实验要求实现矩阵的 QR 分解,并使用基本的 QR
算法尝试计算给定矩阵的所有特征值,观察算法的收敛过程并给出解释。

\subsubsection{实验过程}\label{ux5b9eux9a8cux8fc7ux7a0b}

首先使用 Householder 旋转实现矩阵的 QR 分解:

    \begin{Verbatim}[commandchars=\\\{\}]
{\color{incolor}In [{\color{incolor}4}]:} \PY{k}{def} \PY{n+nf}{householder}\PY{p}{(}\PY{n}{x}\PY{p}{)}\PY{p}{:}
            \PY{k}{if} \PY{p}{(}\PY{n}{x}\PY{p}{[}\PY{l+m+mi}{0}\PY{p}{]} \PY{o}{\PYZgt{}}\PY{o}{=} \PY{l+m+mi}{0}\PY{p}{)}\PY{p}{:}
                \PY{n}{sign} \PY{o}{=} \PY{l+m+mi}{1}
            \PY{k}{else}\PY{p}{:}
                \PY{n}{sign} \PY{o}{=} \PY{o}{\PYZhy{}}\PY{l+m+mi}{1}
            \PY{n}{sigma} \PY{o}{=} \PY{n}{sign} \PY{o}{*} \PY{n}{np}\PY{o}{.}\PY{n}{linalg}\PY{o}{.}\PY{n}{norm}\PY{p}{(}\PY{n}{x}\PY{p}{,} \PY{n+nb}{ord}\PY{o}{=}\PY{l+m+mi}{2}\PY{p}{)}
            \PY{k}{if} \PY{n}{np}\PY{o}{.}\PY{n}{abs}\PY{p}{(}\PY{n}{sigma} \PY{o}{\PYZhy{}} \PY{n}{x}\PY{p}{[}\PY{l+m+mi}{0}\PY{p}{]}\PY{p}{)} \PY{o}{\PYZlt{}} \PY{l+m+mf}{1e\PYZhy{}10}\PY{p}{:}
                \PY{k}{return} \PY{k+kc}{None}
            \PY{n}{h} \PY{o}{=} \PY{n}{x}\PY{o}{.}\PY{n}{copy}\PY{p}{(}\PY{p}{)}
            \PY{n}{h}\PY{p}{[}\PY{l+m+mi}{0}\PY{p}{]} \PY{o}{+}\PY{o}{=} \PY{n}{sigma}
            \PY{k}{return} \PY{n}{h}
            
        
        \PY{k}{def} \PY{n+nf}{QR}\PY{p}{(}\PY{n}{A}\PY{p}{)}\PY{p}{:}
            \PY{n}{n} \PY{o}{=} \PY{n}{A}\PY{o}{.}\PY{n}{shape}\PY{p}{[}\PY{l+m+mi}{0}\PY{p}{]}
            \PY{n}{R} \PY{o}{=} \PY{n}{A}\PY{o}{.}\PY{n}{copy}\PY{p}{(}\PY{p}{)}
            \PY{n}{Q} \PY{o}{=} \PY{n}{np}\PY{o}{.}\PY{n}{identity}\PY{p}{(}\PY{n}{n}\PY{p}{)}
            \PY{k}{for} \PY{n}{i} \PY{o+ow}{in} \PY{n+nb}{range}\PY{p}{(}\PY{n}{n} \PY{o}{\PYZhy{}} \PY{l+m+mi}{1}\PY{p}{)}\PY{p}{:}
                \PY{c+c1}{\PYZsh{} get sub\PYZhy{}matrix}
                \PY{n}{R\PYZus{}1} \PY{o}{=} \PY{n}{R}\PY{p}{[}\PY{n}{i}\PY{p}{:}\PY{p}{,}\PY{n}{i}\PY{p}{:}\PY{p}{]}
                \PY{c+c1}{\PYZsh{} householder vector v and w}
                \PY{n}{v} \PY{o}{=} \PY{n}{householder}\PY{p}{(}\PY{n}{R\PYZus{}1}\PY{p}{[}\PY{p}{:}\PY{p}{,}\PY{l+m+mi}{0}\PY{p}{]}\PY{p}{)}
                \PY{k}{if} \PY{n}{v} \PY{o+ow}{is} \PY{k+kc}{None}\PY{p}{:} \PY{c+c1}{\PYZsh{} go to next submatrix}
                    \PY{k}{continue}
                \PY{n}{w} \PY{o}{=} \PY{n}{v} \PY{o}{/} \PY{n}{np}\PY{o}{.}\PY{n}{linalg}\PY{o}{.}\PY{n}{norm}\PY{p}{(}\PY{n}{v}\PY{p}{,} \PY{n+nb}{ord}\PY{o}{=}\PY{l+m+mi}{2}\PY{p}{)}
                \PY{c+c1}{\PYZsh{} caculate H and transform Q}
                \PY{n}{H} \PY{o}{=} \PY{n}{np}\PY{o}{.}\PY{n}{identity}\PY{p}{(}\PY{n}{n}\PY{p}{)}
                \PY{n}{H}\PY{p}{[}\PY{n}{i}\PY{p}{:}\PY{p}{,}\PY{n}{i}\PY{p}{:}\PY{p}{]} \PY{o}{=} \PY{n}{np}\PY{o}{.}\PY{n}{identity}\PY{p}{(}\PY{n}{n} \PY{o}{\PYZhy{}} \PY{n}{i}\PY{p}{)} \PY{o}{\PYZhy{}} \PY{l+m+mi}{2} \PY{o}{*} \PY{n}{np}\PY{o}{.}\PY{n}{dot}\PY{p}{(}\PY{n}{w}\PY{p}{,} \PY{n}{w}\PY{o}{.}\PY{n}{transpose}\PY{p}{(}\PY{p}{)}\PY{p}{)}
                \PY{n}{Q} \PY{o}{=} \PY{n}{np}\PY{o}{.}\PY{n}{matmul}\PY{p}{(}\PY{n}{Q}\PY{p}{,} \PY{n}{H}\PY{p}{)}
                \PY{c+c1}{\PYZsh{} use v to calculate transformed R}
                \PY{n}{beta} \PY{o}{=} \PY{n}{np}\PY{o}{.}\PY{n}{dot}\PY{p}{(}\PY{n}{v}\PY{o}{.}\PY{n}{transpose}\PY{p}{(}\PY{p}{)}\PY{p}{,} \PY{n}{v}\PY{p}{)}\PY{p}{[}\PY{l+m+mi}{0}\PY{p}{,}\PY{l+m+mi}{0}\PY{p}{]}
                \PY{k}{for} \PY{n}{j} \PY{o+ow}{in} \PY{n+nb}{range}\PY{p}{(}\PY{n}{n} \PY{o}{\PYZhy{}} \PY{n}{i}\PY{p}{)}\PY{p}{:}
                    \PY{n}{gamma} \PY{o}{=} \PY{n}{np}\PY{o}{.}\PY{n}{dot}\PY{p}{(}\PY{n}{v}\PY{o}{.}\PY{n}{transpose}\PY{p}{(}\PY{p}{)}\PY{p}{,} \PY{n}{R\PYZus{}1}\PY{p}{[}\PY{p}{:}\PY{p}{,}\PY{n}{j}\PY{p}{]}\PY{p}{)}\PY{p}{[}\PY{l+m+mi}{0}\PY{p}{,}\PY{l+m+mi}{0}\PY{p}{]}
                    \PY{n}{R\PYZus{}1}\PY{p}{[}\PY{p}{:}\PY{p}{,}\PY{n}{j}\PY{p}{]} \PY{o}{\PYZhy{}}\PY{o}{=} \PY{l+m+mi}{2} \PY{o}{*} \PY{n}{gamma} \PY{o}{/} \PY{n}{beta} \PY{o}{*} \PY{n}{v}
            \PY{k}{return} \PY{n}{Q}\PY{p}{,} \PY{n}{R}
\end{Verbatim}


    使用 numpy 内置的 QR 分解可以测试算法的正确性:

    \begin{Verbatim}[commandchars=\\\{\}]
{\color{incolor}In [{\color{incolor}5}]:} \PY{n}{A} \PY{o}{=} \PY{n}{np}\PY{o}{.}\PY{n}{matrix}\PY{p}{(}\PY{p}{[}\PY{p}{[}\PY{l+m+mf}{1.}\PY{p}{,} \PY{l+m+mi}{2}\PY{p}{]}\PY{p}{,} \PY{p}{[}\PY{l+m+mi}{3}\PY{p}{,} \PY{l+m+mi}{4}\PY{p}{]}\PY{p}{]}\PY{p}{)}
        \PY{n}{Q}\PY{p}{,} \PY{n}{R} \PY{o}{=} \PY{n}{QR}\PY{p}{(}\PY{n}{A}\PY{p}{)}
        \PY{n}{Q\PYZus{}n}\PY{p}{,} \PY{n}{R\PYZus{}n} \PY{o}{=} \PY{n}{np}\PY{o}{.}\PY{n}{linalg}\PY{o}{.}\PY{n}{qr}\PY{p}{(}\PY{n}{A}\PY{p}{)}
        \PY{n+nb}{print}\PY{p}{(}\PY{l+s+s1}{\PYZsq{}}\PY{l+s+s1}{Total Error: }\PY{l+s+si}{\PYZob{}:.3e\PYZcb{}}\PY{l+s+s1}{, Q Error: }\PY{l+s+si}{\PYZob{}:.3e\PYZcb{}}\PY{l+s+s1}{,R Error: }\PY{l+s+si}{\PYZob{}:.3e\PYZcb{}}\PY{l+s+s1}{\PYZsq{}}\PY{o}{.}\PY{n}{format}\PY{p}{(}\PY{n}{np}\PY{o}{.}\PY{n}{max}\PY{p}{(}\PY{n}{np}\PY{o}{.}\PY{n}{abs}\PY{p}{(}\PY{n}{Q} \PY{o}{*} \PY{n}{R} \PY{o}{\PYZhy{}} \PY{n}{A}\PY{p}{)}\PY{p}{)}\PY{p}{,} \PY{n}{np}\PY{o}{.}\PY{n}{max}\PY{p}{(}\PY{n}{np}\PY{o}{.}\PY{n}{abs}\PY{p}{(}\PY{n}{Q} \PY{o}{\PYZhy{}} \PY{n}{Q\PYZus{}n}\PY{p}{)}\PY{p}{)}\PY{p}{,} \PY{n}{np}\PY{o}{.}\PY{n}{max}\PY{p}{(}\PY{n}{np}\PY{o}{.}\PY{n}{abs}\PY{p}{(}\PY{n}{R} \PY{o}{\PYZhy{}} \PY{n}{R\PYZus{}n}\PY{p}{)}\PY{p}{)}\PY{p}{)}\PY{p}{)}
\end{Verbatim}


    \begin{Verbatim}[commandchars=\\\{\}]
Total Error: 9.992e-16, Q Error: 2.220e-16,R Error: 8.882e-16

    \end{Verbatim}

    接下来可以实现基本的 QR
算法求特征值。在判定拟对角阵和求解拟对角阵的特征值时,都需要对对角块为 2
* 2 矩阵的情况进行特殊处理。

    \begin{Verbatim}[commandchars=\\\{\}]
{\color{incolor}In [{\color{incolor}6}]:} \PY{n}{eps} \PY{o}{=} \PY{l+m+mf}{1e\PYZhy{}5}
        
        \PY{c+c1}{\PYZsh{} check if the matrix is quasi\PYZhy{}diagonal}
        \PY{k}{def} \PY{n+nf}{check\PYZus{}quasi\PYZus{}diag}\PY{p}{(}\PY{n}{A}\PY{p}{)}\PY{p}{:}
            \PY{n}{n} \PY{o}{=} \PY{n}{A}\PY{o}{.}\PY{n}{shape}\PY{p}{[}\PY{l+m+mi}{0}\PY{p}{]}
            \PY{n}{cond} \PY{o}{=} \PY{n}{A} \PY{o}{\PYZlt{}} \PY{n}{eps}
            \PY{n}{i} \PY{o}{=} \PY{l+m+mi}{0}
            \PY{k}{while} \PY{n}{i} \PY{o}{\PYZlt{}} \PY{n}{n}\PY{p}{:}
                \PY{n}{cond}\PY{p}{[}\PY{n}{i}\PY{p}{,} \PY{n}{i}\PY{p}{]} \PY{o}{=} \PY{k+kc}{True}
                \PY{k}{if} \PY{n}{i} \PY{o}{\PYZlt{}} \PY{n}{n} \PY{o}{\PYZhy{}} \PY{l+m+mi}{1} \PY{o+ow}{and} \PY{n}{cond}\PY{p}{[}\PY{n}{i} \PY{o}{+} \PY{l+m+mi}{1}\PY{p}{,}\PY{n}{i}\PY{p}{]} \PY{o}{==} \PY{k+kc}{False}\PY{p}{:}
                    \PY{c+c1}{\PYZsh{} 2d\PYZhy{}matrix}
                    \PY{n}{cond}\PY{p}{[}\PY{n}{i} \PY{o}{+} \PY{l+m+mi}{1}\PY{p}{,} \PY{n}{i}\PY{p}{]} \PY{o}{=} \PY{k+kc}{True}
                    \PY{n}{i} \PY{o}{+}\PY{o}{=} \PY{l+m+mi}{2}
                \PY{k}{else}\PY{p}{:}
                    \PY{n}{i} \PY{o}{+}\PY{o}{=} \PY{l+m+mi}{1}
            \PY{k}{for} \PY{n}{i} \PY{o+ow}{in} \PY{n+nb}{range}\PY{p}{(}\PY{n}{n}\PY{p}{)}\PY{p}{:}
                \PY{k}{for} \PY{n}{j} \PY{o+ow}{in} \PY{n+nb}{range}\PY{p}{(}\PY{n}{i}\PY{p}{)}\PY{p}{:}
                    \PY{k}{if} \PY{o+ow}{not} \PY{n}{cond}\PY{p}{[}\PY{n}{i}\PY{p}{,} \PY{n}{j}\PY{p}{]}\PY{p}{:}
                        \PY{k}{return} \PY{k+kc}{False}
            \PY{k}{return} \PY{k+kc}{True}
        
        \PY{c+c1}{\PYZsh{} find eigenvalues by each block}
        \PY{k}{def} \PY{n+nf}{derive\PYZus{}eigen}\PY{p}{(}\PY{n}{A}\PY{p}{)}\PY{p}{:}
            \PY{n}{n} \PY{o}{=} \PY{n}{A}\PY{o}{.}\PY{n}{shape}\PY{p}{[}\PY{l+m+mi}{0}\PY{p}{]}
            \PY{n}{eigen} \PY{o}{=} \PY{n}{np}\PY{o}{.}\PY{n}{zeros}\PY{p}{(}\PY{n}{n}\PY{p}{,}\PY{n}{dtype}\PY{o}{=}\PY{n}{np}\PY{o}{.}\PY{n}{complex128}\PY{p}{)}
            \PY{n}{i} \PY{o}{=} \PY{l+m+mi}{0}
            \PY{k}{while} \PY{n}{i} \PY{o}{\PYZlt{}} \PY{n}{n}\PY{p}{:}
                \PY{k}{if} \PY{n}{i} \PY{o}{\PYZlt{}} \PY{n}{n} \PY{o}{\PYZhy{}} \PY{l+m+mi}{1} \PY{o+ow}{and} \PY{n}{A}\PY{p}{[}\PY{n}{i} \PY{o}{+} \PY{l+m+mi}{1}\PY{p}{,} \PY{n}{i}\PY{p}{]} \PY{o}{\PYZgt{}} \PY{n}{eps}\PY{p}{:}
                    \PY{c+c1}{\PYZsh{} 2d\PYZhy{}matrix}
                    \PY{n}{eigen}\PY{p}{[}\PY{n}{i} \PY{p}{:} \PY{n}{i} \PY{o}{+} \PY{l+m+mi}{2}\PY{p}{]} \PY{o}{=} \PY{n}{np}\PY{o}{.}\PY{n}{linalg}\PY{o}{.}\PY{n}{eig}\PY{p}{(}\PY{n}{A}\PY{p}{[}\PY{n}{i}\PY{p}{:}\PY{n}{i}\PY{o}{+}\PY{l+m+mi}{2}\PY{p}{,}\PY{n}{i}\PY{p}{:}\PY{n}{i}\PY{o}{+}\PY{l+m+mi}{2}\PY{p}{]}\PY{p}{)}\PY{p}{[}\PY{l+m+mi}{0}\PY{p}{]}
                    \PY{n}{i} \PY{o}{+}\PY{o}{=} \PY{l+m+mi}{2}
                \PY{k}{else}\PY{p}{:}
                    \PY{c+c1}{\PYZsh{} 1d\PYZhy{}matrix}
                    \PY{n}{eigen}\PY{p}{[}\PY{n}{i}\PY{p}{]} \PY{o}{=} \PY{n}{A}\PY{p}{[}\PY{n}{i}\PY{p}{,} \PY{n}{i}\PY{p}{]}
                    \PY{n}{i} \PY{o}{+}\PY{o}{=} \PY{l+m+mi}{1}
            \PY{k}{return} \PY{n}{np}\PY{o}{.}\PY{n}{round}\PY{p}{(}\PY{n}{eigen}\PY{p}{,} \PY{n}{decimals}\PY{o}{=}\PY{l+m+mi}{4}\PY{p}{)}
        
        \PY{c+c1}{\PYZsh{} basic QR algorithm}
        \PY{k}{def} \PY{n+nf}{QR\PYZus{}eigen}\PY{p}{(}\PY{n}{A}\PY{p}{)}\PY{p}{:}
            \PY{n}{n} \PY{o}{=} \PY{n}{A}\PY{o}{.}\PY{n}{shape}\PY{p}{[}\PY{l+m+mi}{0}\PY{p}{]}
            \PY{n}{step} \PY{o}{=} \PY{l+m+mi}{0}
            \PY{k}{while} \PY{o+ow}{not} \PY{n}{check\PYZus{}quasi\PYZus{}diag}\PY{p}{(}\PY{n}{A}\PY{p}{)}\PY{p}{:}
                \PY{c+c1}{\PYZsh{} iterate}
                \PY{n}{Q}\PY{p}{,} \PY{n}{R} \PY{o}{=} \PY{n}{np}\PY{o}{.}\PY{n}{linalg}\PY{o}{.}\PY{n}{qr}\PY{p}{(}\PY{n}{A}\PY{p}{)}
                \PY{n}{A\PYZus{}new} \PY{o}{=} \PY{n}{R} \PY{o}{*} \PY{n}{Q}
                \PY{n}{step} \PY{o}{+}\PY{o}{=} \PY{l+m+mi}{1}
                \PY{c+c1}{\PYZsh{} iteration converged}
                \PY{k}{if} \PY{n}{np}\PY{o}{.}\PY{n}{max}\PY{p}{(}\PY{n}{np}\PY{o}{.}\PY{n}{abs}\PY{p}{(}\PY{n}{A\PYZus{}new} \PY{o}{\PYZhy{}} \PY{n}{A}\PY{p}{)}\PY{p}{)} \PY{o}{\PYZlt{}} \PY{l+m+mf}{1e\PYZhy{}8}\PY{p}{:}
                    \PY{n+nb}{print}\PY{p}{(}\PY{l+s+s1}{\PYZsq{}}\PY{l+s+s1}{QR algorithm converged to non\PYZhy{}quasi\PYZhy{}diagonal matrix after }\PY{l+s+si}{\PYZob{}\PYZcb{}}\PY{l+s+s1}{ steps, failed to find eigenvalues}\PY{l+s+s1}{\PYZsq{}}\PY{o}{.}\PY{n}{format}\PY{p}{(}\PY{n}{step}\PY{p}{)}\PY{p}{)}
                    \PY{k}{return} \PY{k+kc}{None}
                \PY{n}{A} \PY{o}{=} \PY{n}{A\PYZus{}new}
            \PY{n+nb}{print}\PY{p}{(}\PY{l+s+s1}{\PYZsq{}}\PY{l+s+s1}{QR algorithm found eigenvalues of A after }\PY{l+s+si}{\PYZob{}\PYZcb{}}\PY{l+s+s1}{ steps}\PY{l+s+s1}{\PYZsq{}}\PY{o}{.}\PY{n}{format}\PY{p}{(}\PY{n}{step}\PY{p}{)}\PY{p}{)}
            \PY{k}{return} \PY{n}{derive\PYZus{}eigen}\PY{p}{(}\PY{n}{A}\PY{p}{)}
\end{Verbatim}


    对题中给出的矩阵使用 QR 算法:

    \begin{Verbatim}[commandchars=\\\{\}]
{\color{incolor}In [{\color{incolor}7}]:} \PY{n}{A} \PY{o}{=} \PY{n}{np}\PY{o}{.}\PY{n}{matrix}\PY{p}{(}\PY{p}{[}\PY{p}{[}\PY{l+m+mf}{0.5}\PY{p}{,} \PY{l+m+mf}{0.5}\PY{p}{,} \PY{l+m+mf}{0.5}\PY{p}{,} \PY{l+m+mf}{0.5}\PY{p}{]}\PY{p}{,} \PY{p}{[}\PY{l+m+mf}{0.5}\PY{p}{,} \PY{l+m+mf}{0.5}\PY{p}{,} \PY{o}{\PYZhy{}}\PY{l+m+mf}{0.5} \PY{p}{,} \PY{o}{\PYZhy{}}\PY{l+m+mf}{0.5}\PY{p}{]}\PY{p}{,} \PY{p}{[}\PY{l+m+mf}{0.5}\PY{p}{,} \PY{o}{\PYZhy{}}\PY{l+m+mf}{0.5}\PY{p}{,} \PY{l+m+mf}{0.5}\PY{p}{,} \PY{o}{\PYZhy{}}\PY{l+m+mf}{0.5}\PY{p}{]}\PY{p}{,} \PY{p}{[}\PY{l+m+mf}{0.5}\PY{p}{,} \PY{o}{\PYZhy{}}\PY{l+m+mf}{0.5}\PY{p}{,} \PY{o}{\PYZhy{}}\PY{l+m+mf}{0.5}\PY{p}{,} \PY{l+m+mf}{0.5}\PY{p}{]}\PY{p}{]}\PY{p}{)}
        \PY{n}{l\PYZus{}A} \PY{o}{=} \PY{n}{QR\PYZus{}eigen}\PY{p}{(}\PY{n}{A}\PY{p}{)}
\end{Verbatim}


    \begin{Verbatim}[commandchars=\\\{\}]
QR algorithm converged to non-quasi-diagonal matrix after 1 steps, failed to find eigenvalues

    \end{Verbatim}

    算法在进行了一步迭代后就失败了,结合代码中的判定条件,可知一步迭代后 A
没有发生变化,如下所示:

    \begin{Verbatim}[commandchars=\\\{\}]
{\color{incolor}In [{\color{incolor}8}]:} \PY{n}{Q}\PY{p}{,} \PY{n}{R} \PY{o}{=} \PY{n}{QR}\PY{p}{(}\PY{n}{A}\PY{p}{)}
        \PY{n}{R} \PY{o}{*} \PY{n}{Q}
\end{Verbatim}


\begin{Verbatim}[commandchars=\\\{\}]
{\color{outcolor}Out[{\color{outcolor}8}]:} matrix([[ 0.5,  0.5,  0.5,  0.5],
                [ 0.5,  0.5, -0.5, -0.5],
                [ 0.5, -0.5,  0.5, -0.5],
                [ 0.5, -0.5, -0.5,  0.5]])
\end{Verbatim}
            
    这是由于 \(\mathbf{A}\) 事实上本身是一个正交矩阵,因此 QR 分解得到的
\(\mathbf{R}\) 是恒等的(或者只差一个符号),故无法使用基本的 QR
算法进行迭代寻找特征值。

    \subsection{上机题 4}\label{ux4e0aux673aux9898-4}

\subsubsection{实验概述}\label{ux5b9eux9a8cux6982ux8ff0}

本题要求用带原点位移的 QR
算法计算第三题中矩阵的特征值,并观察收敛结果,与第三题进行比较。

\subsubsection{实验过程}\label{ux5b9eux9a8cux8fc7ux7a0b}

实现带原点位移的 QR
算法,并打印每次迭代过程。当每次迭代出一个特征值后,都检查矩阵的(拟)对角性;如果成立则立刻停止迭代。

    \begin{Verbatim}[commandchars=\\\{\}]
{\color{incolor}In [{\color{incolor}9}]:} \PY{k}{def} \PY{n+nf}{print\PYZus{}matrix}\PY{p}{(}\PY{n}{A}\PY{p}{)}\PY{p}{:}
            \PY{n}{n} \PY{o}{=} \PY{n}{A}\PY{o}{.}\PY{n}{shape}\PY{p}{[}\PY{l+m+mi}{0}\PY{p}{]}
            \PY{k}{for} \PY{n}{i} \PY{o+ow}{in} \PY{n+nb}{range}\PY{p}{(}\PY{n}{n}\PY{p}{)}\PY{p}{:}
                \PY{n+nb}{print}\PY{p}{(}\PY{l+s+s1}{\PYZsq{}}\PY{l+s+se}{\PYZbs{}t}\PY{l+s+s1}{\PYZsq{}}\PY{o}{.}\PY{n}{join}\PY{p}{(}\PY{n+nb}{map}\PY{p}{(}\PY{k}{lambda} \PY{n}{x}\PY{p}{:} \PY{l+s+s1}{\PYZsq{}}\PY{l+s+si}{\PYZob{}: .4f\PYZcb{}}\PY{l+s+s1}{\PYZsq{}}\PY{o}{.}\PY{n}{format}\PY{p}{(}\PY{n}{x}\PY{p}{)}\PY{p}{,} \PY{n}{A}\PY{p}{[}\PY{n}{i}\PY{p}{,}\PY{p}{:}\PY{p}{]}\PY{o}{.}\PY{n}{tolist}\PY{p}{(}\PY{p}{)}\PY{p}{[}\PY{l+m+mi}{0}\PY{p}{]}\PY{p}{)}\PY{p}{)}\PY{p}{)}
                    
        
        \PY{k}{def} \PY{n+nf}{QR\PYZus{}shift\PYZus{}eigen}\PY{p}{(}\PY{n}{A}\PY{p}{,} \PY{n}{n}\PY{o}{=}\PY{k+kc}{None}\PY{p}{)}\PY{p}{:}
            \PY{k}{if} \PY{n}{n} \PY{o+ow}{is} \PY{k+kc}{None}\PY{p}{:} \PY{c+c1}{\PYZsh{} initial calling}
                \PY{n}{n} \PY{o}{=} \PY{n}{A}\PY{o}{.}\PY{n}{shape}\PY{p}{[}\PY{l+m+mi}{0}\PY{p}{]}
                \PY{n}{A} \PY{o}{=} \PY{n}{A}\PY{o}{.}\PY{n}{copy}\PY{p}{(}\PY{p}{)}
                \PY{n+nb}{print}\PY{p}{(}\PY{l+s+s1}{\PYZsq{}}\PY{l+s+s1}{Original matrix:}\PY{l+s+s1}{\PYZsq{}}\PY{p}{)}
                \PY{n}{print\PYZus{}matrix}\PY{p}{(}\PY{n}{A}\PY{p}{)}
            \PY{k}{if} \PY{n}{n} \PY{o}{\PYZlt{}}\PY{o}{=} \PY{l+m+mi}{1} \PY{o+ow}{or} \PY{n}{check\PYZus{}quasi\PYZus{}diag}\PY{p}{(}\PY{n}{A}\PY{p}{)}\PY{p}{:}
                \PY{n+nb}{print}\PY{p}{(}\PY{l+s+s1}{\PYZsq{}}\PY{l+s+s1}{Matrix is already quasi\PYZhy{}diagonal, end iteration}\PY{l+s+s1}{\PYZsq{}}\PY{p}{)}
                \PY{k}{return}
            \PY{c+c1}{\PYZsh{} find the last diagonal element of size n}
            \PY{n}{count} \PY{o}{=} \PY{l+m+mi}{0}
            \PY{k}{while} \PY{n}{np}\PY{o}{.}\PY{n}{abs}\PY{p}{(}\PY{n}{A}\PY{p}{[}\PY{n}{n} \PY{o}{\PYZhy{}} \PY{l+m+mi}{1}\PY{p}{,}\PY{n}{n} \PY{o}{\PYZhy{}} \PY{l+m+mi}{2}\PY{p}{]}\PY{p}{)} \PY{o}{\PYZgt{}} \PY{n}{eps} \PY{o+ow}{or} \PY{n}{np}\PY{o}{.}\PY{n}{abs}\PY{p}{(}\PY{n}{A}\PY{p}{[}\PY{n}{n} \PY{o}{\PYZhy{}} \PY{l+m+mi}{1}\PY{p}{,}\PY{n}{n} \PY{o}{\PYZhy{}} \PY{l+m+mi}{1}\PY{p}{]}\PY{p}{)} \PY{o}{\PYZlt{}} \PY{n}{eps}\PY{p}{:}
                \PY{n}{old\PYZus{}A} \PY{o}{=} \PY{n}{A}\PY{o}{.}\PY{n}{copy}\PY{p}{(}\PY{p}{)}
                \PY{n}{s} \PY{o}{=} \PY{n}{A}\PY{p}{[}\PY{n}{n} \PY{o}{\PYZhy{}} \PY{l+m+mi}{1}\PY{p}{,} \PY{n}{n} \PY{o}{\PYZhy{}} \PY{l+m+mi}{1}\PY{p}{]}
                \PY{n}{Q}\PY{p}{,} \PY{n}{R} \PY{o}{=} \PY{n}{QR}\PY{p}{(}\PY{n}{A}\PY{p}{[}\PY{p}{:}\PY{n}{n}\PY{p}{,}\PY{p}{:}\PY{n}{n}\PY{p}{]} \PY{o}{\PYZhy{}} \PY{n}{s} \PY{o}{*} \PY{n}{np}\PY{o}{.}\PY{n}{identity}\PY{p}{(}\PY{n}{n}\PY{p}{)}\PY{p}{)}
                \PY{n}{A}\PY{p}{[}\PY{p}{:}\PY{n}{n}\PY{p}{,}\PY{p}{:}\PY{n}{n}\PY{p}{]} \PY{o}{=} \PY{n}{R} \PY{o}{*} \PY{n}{Q} \PY{o}{+} \PY{n}{s} \PY{o}{*} \PY{n}{np}\PY{o}{.}\PY{n}{identity}\PY{p}{(}\PY{n}{n}\PY{p}{)}
                \PY{n}{count} \PY{o}{+}\PY{o}{=} \PY{l+m+mi}{1}
                \PY{n+nb}{print}\PY{p}{(}\PY{l+s+s1}{\PYZsq{}}\PY{l+s+s1}{After iteration }\PY{l+s+si}{\PYZob{}\PYZcb{}}\PY{l+s+s1}{:}\PY{l+s+s1}{\PYZsq{}}\PY{o}{.}\PY{n}{format}\PY{p}{(}\PY{n}{count}\PY{p}{)}\PY{p}{)}
                \PY{n}{print\PYZus{}matrix}\PY{p}{(}\PY{n}{A}\PY{p}{)}
                \PY{k}{if} \PY{n}{np}\PY{o}{.}\PY{n}{max}\PY{p}{(}\PY{n}{np}\PY{o}{.}\PY{n}{abs}\PY{p}{(}\PY{n}{A} \PY{o}{\PYZhy{}} \PY{n}{old\PYZus{}A}\PY{p}{)}\PY{p}{)} \PY{o}{\PYZlt{}} \PY{n}{eps}\PY{p}{:}
                    \PY{k}{raise} \PY{n+ne}{Exception}\PY{p}{(}\PY{l+s+s1}{\PYZsq{}}\PY{l+s+s1}{Iteration converged but no more eigenvalue is found}\PY{l+s+s1}{\PYZsq{}}\PY{p}{)}
            \PY{n+nb}{print}\PY{p}{(}\PY{l+s+s1}{\PYZsq{}}\PY{l+s+s1}{Shifted QR took }\PY{l+s+si}{\PYZob{}\PYZcb{}}\PY{l+s+s1}{ steps to find eigenvalue }\PY{l+s+si}{\PYZob{}:.4f\PYZcb{}}\PY{l+s+s1}{ of A}\PY{l+s+s1}{\PYZsq{}}\PY{o}{.}\PY{n}{format}\PY{p}{(}\PY{n}{count}\PY{p}{,} \PY{n}{A}\PY{p}{[}\PY{n}{n} \PY{o}{\PYZhy{}} \PY{l+m+mi}{1}\PY{p}{,}\PY{n}{n} \PY{o}{\PYZhy{}} \PY{l+m+mi}{1}\PY{p}{]}\PY{p}{)}\PY{p}{)}
            \PY{n}{QR\PYZus{}shift\PYZus{}eigen}\PY{p}{(}\PY{n}{A}\PY{p}{,} \PY{n}{n} \PY{o}{\PYZhy{}} \PY{l+m+mi}{1}\PY{p}{)}
            \PY{k}{return} \PY{n}{derive\PYZus{}eigen}\PY{p}{(}\PY{n}{A}\PY{p}{)}
\end{Verbatim}


    \begin{Verbatim}[commandchars=\\\{\}]
{\color{incolor}In [{\color{incolor}10}]:} \PY{n}{a\PYZus{}l\PYZus{}shift} \PY{o}{=} \PY{n}{QR\PYZus{}shift\PYZus{}eigen}\PY{p}{(}\PY{n}{A}\PY{p}{)}
\end{Verbatim}


    \begin{Verbatim}[commandchars=\\\{\}]
Original matrix:
 0.5000	 0.5000	 0.5000	 0.5000
 0.5000	 0.5000	-0.5000	-0.5000
 0.5000	-0.5000	 0.5000	-0.5000
 0.5000	-0.5000	-0.5000	 0.5000
After iteration 1:
-0.5000	 0.6708	-0.4392	-0.3273
 0.6708	 0.7000	 0.1964	 0.1464
-0.4392	 0.1964	 0.8714	-0.0958
-0.3273	 0.1464	-0.0958	 0.9286
After iteration 2:
-0.9991	-0.0349	 0.0202	-0.0143
-0.0349	 0.9994	 0.0004	-0.0002
 0.0202	 0.0004	 0.9998	 0.0001
-0.0143	-0.0002	 0.0001	 0.9999
After iteration 3:
-1.0000	-0.0000	 0.0000	-0.0000
-0.0000	 1.0000	 0.0000	-0.0000
 0.0000	 0.0000	 1.0000	 0.0000
-0.0000	-0.0000	 0.0000	 1.0000
Shifted QR took 3 steps to find eigenvalue 1.0000 of A
Matrix is already quasi-diagonal, end iteration

    \end{Verbatim}

    可以看到,带原点位移的 QR 算法解决了简单 QR
算法处理正交矩阵时的问题(因为位移破坏了正交性),仅在三个迭代后就得到了第一个特征值。并且此时矩阵刚好已成为对角矩阵,故所有特征值都已经找到:

    \begin{Verbatim}[commandchars=\\\{\}]
{\color{incolor}In [{\color{incolor}11}]:} \PY{n}{a\PYZus{}l\PYZus{}shift}
\end{Verbatim}


\begin{Verbatim}[commandchars=\\\{\}]
{\color{outcolor}Out[{\color{outcolor}11}]:} array([-1.+0.j,  1.+0.j,  1.+0.j,  1.+0.j])
\end{Verbatim}
            
    我们还可以使用更多的正交矩阵进行测试,比如下列矩阵有一对共轭复特征值:

    \begin{Verbatim}[commandchars=\\\{\}]
{\color{incolor}In [{\color{incolor}12}]:} \PY{n}{A} \PY{o}{=} \PY{n}{np}\PY{o}{.}\PY{n}{matrix}\PY{p}{(}\PY{p}{[}\PY{p}{[}\PY{l+m+mi}{0}\PY{p}{,} \PY{o}{\PYZhy{}}\PY{l+m+mf}{0.8}\PY{p}{,} \PY{o}{\PYZhy{}}\PY{l+m+mf}{0.6}\PY{p}{]}\PY{p}{,} \PY{p}{[}\PY{l+m+mf}{0.8}\PY{p}{,} \PY{o}{\PYZhy{}}\PY{l+m+mf}{0.36}\PY{p}{,} \PY{l+m+mf}{0.48}\PY{p}{]}\PY{p}{,} \PY{p}{[}\PY{l+m+mf}{0.6}\PY{p}{,} \PY{l+m+mf}{0.48}\PY{p}{,} \PY{o}{\PYZhy{}}\PY{l+m+mf}{0.64}\PY{p}{]}\PY{p}{]}\PY{p}{)}
         \PY{n}{QR\PYZus{}shift\PYZus{}eigen}\PY{p}{(}\PY{n}{A}\PY{p}{)}
\end{Verbatim}


    \begin{Verbatim}[commandchars=\\\{\}]
Original matrix:
 0.0000	-0.8000	-0.6000
 0.8000	-0.3600	 0.4800
 0.6000	 0.4800	-0.6400
After iteration 1:
 0.0000	-0.9751	-0.2218
 0.9751	-0.0492	 0.2162
 0.2218	 0.2162	-0.9508
After iteration 2:
 0.0000	-1.0000	-0.0081
 1.0000	-0.0001	 0.0081
 0.0081	 0.0081	-0.9999
After iteration 3:
-0.0000	-1.0000	-0.0000
 1.0000	-0.0000	 0.0000
 0.0000	 0.0000	-1.0000
Shifted QR took 3 steps to find eigenvalue -1.0000 of A
Matrix is already quasi-diagonal, end iteration

    \end{Verbatim}

\begin{Verbatim}[commandchars=\\\{\}]
{\color{outcolor}Out[{\color{outcolor}12}]:} array([-0.+1.j, -0.-1.j, -1.+0.j])
\end{Verbatim}
            
    可以看到带原点位移的 QR
算法也顺利地将其迭代成为拟对角矩阵,并且找到了所有的特征值。

    但是书中给出的单位移策略也并非通用的,例如对于下列矩阵,这一策略就无法求得特征值:

    \begin{Verbatim}[commandchars=\\\{\}]
{\color{incolor}In [{\color{incolor}13}]:} \PY{n}{A} \PY{o}{=} \PY{n}{np}\PY{o}{.}\PY{n}{matrix}\PY{p}{(}\PY{p}{[}\PY{p}{[}\PY{l+m+mf}{0.}\PY{p}{,} \PY{l+m+mi}{0}\PY{p}{,} \PY{l+m+mi}{0}\PY{p}{,} \PY{l+m+mi}{1}\PY{p}{]}\PY{p}{,} \PY{p}{[}\PY{l+m+mi}{0}\PY{p}{,} \PY{l+m+mi}{0}\PY{p}{,} \PY{l+m+mi}{1}\PY{p}{,} \PY{l+m+mi}{0}\PY{p}{]}\PY{p}{,} \PY{p}{[}\PY{l+m+mi}{0}\PY{p}{,} \PY{l+m+mi}{1}\PY{p}{,} \PY{l+m+mi}{0}\PY{p}{,} \PY{l+m+mi}{0}\PY{p}{]}\PY{p}{,} \PY{p}{[}\PY{l+m+mi}{1}\PY{p}{,} \PY{l+m+mi}{0}\PY{p}{,} \PY{l+m+mi}{0}\PY{p}{,} \PY{l+m+mi}{0}\PY{p}{]}\PY{p}{]}\PY{p}{)}
         \PY{k}{try}\PY{p}{:}
             \PY{n}{QR\PYZus{}shift\PYZus{}eigen}\PY{p}{(}\PY{n}{A}\PY{p}{)}
         \PY{k}{except} \PY{n+ne}{Exception} \PY{k}{as} \PY{n}{e}\PY{p}{:}
             \PY{n+nb}{print}\PY{p}{(}\PY{n}{e}\PY{p}{)}
\end{Verbatim}


    \begin{Verbatim}[commandchars=\\\{\}]
Original matrix:
 0.0000	 0.0000	 0.0000	 1.0000
 0.0000	 0.0000	 1.0000	 0.0000
 0.0000	 1.0000	 0.0000	 0.0000
 1.0000	 0.0000	 0.0000	 0.0000
After iteration 1:
-0.0000	 0.0000	 0.0000	 1.0000
 0.0000	-0.0000	 1.0000	 0.0000
 0.0000	 1.0000	-0.0000	 0.0000
 1.0000	 0.0000	 0.0000	-0.0000
Iteration converged but no more eigenvalue is found

    \end{Verbatim}

    \subsection{实验结论}\label{ux5b9eux9a8cux7ed3ux8bba}

本实验中,我实现了求矩阵特征值的三种方法。幂法较为简单,可以快速求出绝对值最大的特征值。简单
QR 算法和带简单原点位移策略的 QR
算法都能求所有特征值,且后者的适用范围更广。事实上,如果使用更佳的策略(如双位移),带原点位移的
QR 算法总是能够收敛到拟三角阵,从而能方便地求出特征值。


    % Add a bibliography block to the postdoc
    
    
    
    \end{document}
