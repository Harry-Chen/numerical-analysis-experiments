
% Default to the notebook output style

    


% Inherit from the specified cell style.




    
\documentclass[11pt]{ctexart}

    
    
    \usepackage[T1]{fontenc}
    % Nicer default font (+ math font) than Computer Modern for most use cases
    \usepackage{mathpazo}

    % Basic figure setup, for now with no caption control since it's done
    % automatically by Pandoc (which extracts ![](path) syntax from Markdown).
    \usepackage{graphicx}
    % We will generate all images so they have a width \maxwidth. This means
    % that they will get their normal width if they fit onto the page, but
    % are scaled down if they would overflow the margins.
    \makeatletter
    \def\maxwidth{\ifdim\Gin@nat@width>\linewidth\linewidth
    \else\Gin@nat@width\fi}
    \makeatother
    \let\Oldincludegraphics\includegraphics
    % Set max figure width to be 80% of text width, for now hardcoded.
    \renewcommand{\includegraphics}[1]{\Oldincludegraphics[width=.8\maxwidth]{#1}}
    % Ensure that by default, figures have no caption (until we provide a
    % proper Figure object with a Caption API and a way to capture that
    % in the conversion process - todo).
    \usepackage{caption}
    \DeclareCaptionLabelFormat{nolabel}{}
    \captionsetup{labelformat=nolabel}

    \usepackage{adjustbox} % Used to constrain images to a maximum size 
    \usepackage{xcolor} % Allow colors to be defined
    \usepackage{enumerate} % Needed for markdown enumerations to work
    \usepackage{geometry} % Used to adjust the document margins
    \usepackage{amsmath} % Equations
    \usepackage{amssymb} % Equations
    \usepackage{textcomp} % defines textquotesingle
    % Hack from http://tex.stackexchange.com/a/47451/13684:
    \AtBeginDocument{%
        \def\PYZsq{\textquotesingle}% Upright quotes in Pygmentized code
    }
    \usepackage{upquote} % Upright quotes for verbatim code
    \usepackage{eurosym} % defines \euro
    \usepackage[mathletters]{ucs} % Extended unicode (utf-8) support
    \usepackage[utf8x]{inputenc} % Allow utf-8 characters in the tex document
    \usepackage{fancyvrb} % verbatim replacement that allows latex
    \usepackage{grffile} % extends the file name processing of package graphics 
                         % to support a larger range 
    % The hyperref package gives us a pdf with properly built
    % internal navigation ('pdf bookmarks' for the table of contents,
    % internal cross-reference links, web links for URLs, etc.)
    \usepackage{hyperref}
    \usepackage{longtable} % longtable support required by pandoc >1.10
    \usepackage{booktabs}  % table support for pandoc > 1.12.2
    \usepackage[inline]{enumitem} % IRkernel/repr support (it uses the enumerate* environment)
    \usepackage[normalem]{ulem} % ulem is needed to support strikethroughs (\sout)
                                % normalem makes italics be italics, not underlines
    

    
    
    % Colors for the hyperref package
    \definecolor{urlcolor}{rgb}{0,.145,.698}
    \definecolor{linkcolor}{rgb}{.71,0.21,0.01}
    \definecolor{citecolor}{rgb}{.12,.54,.11}

    % ANSI colors
    \definecolor{ansi-black}{HTML}{3E424D}
    \definecolor{ansi-black-intense}{HTML}{282C36}
    \definecolor{ansi-red}{HTML}{E75C58}
    \definecolor{ansi-red-intense}{HTML}{B22B31}
    \definecolor{ansi-green}{HTML}{00A250}
    \definecolor{ansi-green-intense}{HTML}{007427}
    \definecolor{ansi-yellow}{HTML}{DDB62B}
    \definecolor{ansi-yellow-intense}{HTML}{B27D12}
    \definecolor{ansi-blue}{HTML}{208FFB}
    \definecolor{ansi-blue-intense}{HTML}{0065CA}
    \definecolor{ansi-magenta}{HTML}{D160C4}
    \definecolor{ansi-magenta-intense}{HTML}{A03196}
    \definecolor{ansi-cyan}{HTML}{60C6C8}
    \definecolor{ansi-cyan-intense}{HTML}{258F8F}
    \definecolor{ansi-white}{HTML}{C5C1B4}
    \definecolor{ansi-white-intense}{HTML}{A1A6B2}

    % commands and environments needed by pandoc snippets
    % extracted from the output of `pandoc -s`
    \providecommand{\tightlist}{%
      \setlength{\itemsep}{0pt}\setlength{\parskip}{0pt}}
    \DefineVerbatimEnvironment{Highlighting}{Verbatim}{commandchars=\\\{\}}
    % Add ',fontsize=\small' for more characters per line
    \newenvironment{Shaded}{}{}
    \newcommand{\KeywordTok}[1]{\textcolor[rgb]{0.00,0.44,0.13}{\textbf{{#1}}}}
    \newcommand{\DataTypeTok}[1]{\textcolor[rgb]{0.56,0.13,0.00}{{#1}}}
    \newcommand{\DecValTok}[1]{\textcolor[rgb]{0.25,0.63,0.44}{{#1}}}
    \newcommand{\BaseNTok}[1]{\textcolor[rgb]{0.25,0.63,0.44}{{#1}}}
    \newcommand{\FloatTok}[1]{\textcolor[rgb]{0.25,0.63,0.44}{{#1}}}
    \newcommand{\CharTok}[1]{\textcolor[rgb]{0.25,0.44,0.63}{{#1}}}
    \newcommand{\StringTok}[1]{\textcolor[rgb]{0.25,0.44,0.63}{{#1}}}
    \newcommand{\CommentTok}[1]{\textcolor[rgb]{0.38,0.63,0.69}{\textit{{#1}}}}
    \newcommand{\OtherTok}[1]{\textcolor[rgb]{0.00,0.44,0.13}{{#1}}}
    \newcommand{\AlertTok}[1]{\textcolor[rgb]{1.00,0.00,0.00}{\textbf{{#1}}}}
    \newcommand{\FunctionTok}[1]{\textcolor[rgb]{0.02,0.16,0.49}{{#1}}}
    \newcommand{\RegionMarkerTok}[1]{{#1}}
    \newcommand{\ErrorTok}[1]{\textcolor[rgb]{1.00,0.00,0.00}{\textbf{{#1}}}}
    \newcommand{\NormalTok}[1]{{#1}}
    
    % Additional commands for more recent versions of Pandoc
    \newcommand{\ConstantTok}[1]{\textcolor[rgb]{0.53,0.00,0.00}{{#1}}}
    \newcommand{\SpecialCharTok}[1]{\textcolor[rgb]{0.25,0.44,0.63}{{#1}}}
    \newcommand{\VerbatimStringTok}[1]{\textcolor[rgb]{0.25,0.44,0.63}{{#1}}}
    \newcommand{\SpecialStringTok}[1]{\textcolor[rgb]{0.73,0.40,0.53}{{#1}}}
    \newcommand{\ImportTok}[1]{{#1}}
    \newcommand{\DocumentationTok}[1]{\textcolor[rgb]{0.73,0.13,0.13}{\textit{{#1}}}}
    \newcommand{\AnnotationTok}[1]{\textcolor[rgb]{0.38,0.63,0.69}{\textbf{\textit{{#1}}}}}
    \newcommand{\CommentVarTok}[1]{\textcolor[rgb]{0.38,0.63,0.69}{\textbf{\textit{{#1}}}}}
    \newcommand{\VariableTok}[1]{\textcolor[rgb]{0.10,0.09,0.49}{{#1}}}
    \newcommand{\ControlFlowTok}[1]{\textcolor[rgb]{0.00,0.44,0.13}{\textbf{{#1}}}}
    \newcommand{\OperatorTok}[1]{\textcolor[rgb]{0.40,0.40,0.40}{{#1}}}
    \newcommand{\BuiltInTok}[1]{{#1}}
    \newcommand{\ExtensionTok}[1]{{#1}}
    \newcommand{\PreprocessorTok}[1]{\textcolor[rgb]{0.74,0.48,0.00}{{#1}}}
    \newcommand{\AttributeTok}[1]{\textcolor[rgb]{0.49,0.56,0.16}{{#1}}}
    \newcommand{\InformationTok}[1]{\textcolor[rgb]{0.38,0.63,0.69}{\textbf{\textit{{#1}}}}}
    \newcommand{\WarningTok}[1]{\textcolor[rgb]{0.38,0.63,0.69}{\textbf{\textit{{#1}}}}}
    
    
    % Define a nice break command that doesn't care if a line doesn't already
    % exist.
    \def\br{\hspace*{\fill} \\* }
    % Math Jax compatability definitions
    \def\gt{>}
    \def\lt{<}
    % Document parameters
    \title{Chapter3}
    
    
    

    % Pygments definitions
    
\makeatletter
\def\PY@reset{\let\PY@it=\relax \let\PY@bf=\relax%
    \let\PY@ul=\relax \let\PY@tc=\relax%
    \let\PY@bc=\relax \let\PY@ff=\relax}
\def\PY@tok#1{\csname PY@tok@#1\endcsname}
\def\PY@toks#1+{\ifx\relax#1\empty\else%
    \PY@tok{#1}\expandafter\PY@toks\fi}
\def\PY@do#1{\PY@bc{\PY@tc{\PY@ul{%
    \PY@it{\PY@bf{\PY@ff{#1}}}}}}}
\def\PY#1#2{\PY@reset\PY@toks#1+\relax+\PY@do{#2}}

\expandafter\def\csname PY@tok@w\endcsname{\def\PY@tc##1{\textcolor[rgb]{0.73,0.73,0.73}{##1}}}
\expandafter\def\csname PY@tok@c\endcsname{\let\PY@it=\textit\def\PY@tc##1{\textcolor[rgb]{0.25,0.50,0.50}{##1}}}
\expandafter\def\csname PY@tok@cp\endcsname{\def\PY@tc##1{\textcolor[rgb]{0.74,0.48,0.00}{##1}}}
\expandafter\def\csname PY@tok@k\endcsname{\let\PY@bf=\textbf\def\PY@tc##1{\textcolor[rgb]{0.00,0.50,0.00}{##1}}}
\expandafter\def\csname PY@tok@kp\endcsname{\def\PY@tc##1{\textcolor[rgb]{0.00,0.50,0.00}{##1}}}
\expandafter\def\csname PY@tok@kt\endcsname{\def\PY@tc##1{\textcolor[rgb]{0.69,0.00,0.25}{##1}}}
\expandafter\def\csname PY@tok@o\endcsname{\def\PY@tc##1{\textcolor[rgb]{0.40,0.40,0.40}{##1}}}
\expandafter\def\csname PY@tok@ow\endcsname{\let\PY@bf=\textbf\def\PY@tc##1{\textcolor[rgb]{0.67,0.13,1.00}{##1}}}
\expandafter\def\csname PY@tok@nb\endcsname{\def\PY@tc##1{\textcolor[rgb]{0.00,0.50,0.00}{##1}}}
\expandafter\def\csname PY@tok@nf\endcsname{\def\PY@tc##1{\textcolor[rgb]{0.00,0.00,1.00}{##1}}}
\expandafter\def\csname PY@tok@nc\endcsname{\let\PY@bf=\textbf\def\PY@tc##1{\textcolor[rgb]{0.00,0.00,1.00}{##1}}}
\expandafter\def\csname PY@tok@nn\endcsname{\let\PY@bf=\textbf\def\PY@tc##1{\textcolor[rgb]{0.00,0.00,1.00}{##1}}}
\expandafter\def\csname PY@tok@ne\endcsname{\let\PY@bf=\textbf\def\PY@tc##1{\textcolor[rgb]{0.82,0.25,0.23}{##1}}}
\expandafter\def\csname PY@tok@nv\endcsname{\def\PY@tc##1{\textcolor[rgb]{0.10,0.09,0.49}{##1}}}
\expandafter\def\csname PY@tok@no\endcsname{\def\PY@tc##1{\textcolor[rgb]{0.53,0.00,0.00}{##1}}}
\expandafter\def\csname PY@tok@nl\endcsname{\def\PY@tc##1{\textcolor[rgb]{0.63,0.63,0.00}{##1}}}
\expandafter\def\csname PY@tok@ni\endcsname{\let\PY@bf=\textbf\def\PY@tc##1{\textcolor[rgb]{0.60,0.60,0.60}{##1}}}
\expandafter\def\csname PY@tok@na\endcsname{\def\PY@tc##1{\textcolor[rgb]{0.49,0.56,0.16}{##1}}}
\expandafter\def\csname PY@tok@nt\endcsname{\let\PY@bf=\textbf\def\PY@tc##1{\textcolor[rgb]{0.00,0.50,0.00}{##1}}}
\expandafter\def\csname PY@tok@nd\endcsname{\def\PY@tc##1{\textcolor[rgb]{0.67,0.13,1.00}{##1}}}
\expandafter\def\csname PY@tok@s\endcsname{\def\PY@tc##1{\textcolor[rgb]{0.73,0.13,0.13}{##1}}}
\expandafter\def\csname PY@tok@sd\endcsname{\let\PY@it=\textit\def\PY@tc##1{\textcolor[rgb]{0.73,0.13,0.13}{##1}}}
\expandafter\def\csname PY@tok@si\endcsname{\let\PY@bf=\textbf\def\PY@tc##1{\textcolor[rgb]{0.73,0.40,0.53}{##1}}}
\expandafter\def\csname PY@tok@se\endcsname{\let\PY@bf=\textbf\def\PY@tc##1{\textcolor[rgb]{0.73,0.40,0.13}{##1}}}
\expandafter\def\csname PY@tok@sr\endcsname{\def\PY@tc##1{\textcolor[rgb]{0.73,0.40,0.53}{##1}}}
\expandafter\def\csname PY@tok@ss\endcsname{\def\PY@tc##1{\textcolor[rgb]{0.10,0.09,0.49}{##1}}}
\expandafter\def\csname PY@tok@sx\endcsname{\def\PY@tc##1{\textcolor[rgb]{0.00,0.50,0.00}{##1}}}
\expandafter\def\csname PY@tok@m\endcsname{\def\PY@tc##1{\textcolor[rgb]{0.40,0.40,0.40}{##1}}}
\expandafter\def\csname PY@tok@gh\endcsname{\let\PY@bf=\textbf\def\PY@tc##1{\textcolor[rgb]{0.00,0.00,0.50}{##1}}}
\expandafter\def\csname PY@tok@gu\endcsname{\let\PY@bf=\textbf\def\PY@tc##1{\textcolor[rgb]{0.50,0.00,0.50}{##1}}}
\expandafter\def\csname PY@tok@gd\endcsname{\def\PY@tc##1{\textcolor[rgb]{0.63,0.00,0.00}{##1}}}
\expandafter\def\csname PY@tok@gi\endcsname{\def\PY@tc##1{\textcolor[rgb]{0.00,0.63,0.00}{##1}}}
\expandafter\def\csname PY@tok@gr\endcsname{\def\PY@tc##1{\textcolor[rgb]{1.00,0.00,0.00}{##1}}}
\expandafter\def\csname PY@tok@ge\endcsname{\let\PY@it=\textit}
\expandafter\def\csname PY@tok@gs\endcsname{\let\PY@bf=\textbf}
\expandafter\def\csname PY@tok@gp\endcsname{\let\PY@bf=\textbf\def\PY@tc##1{\textcolor[rgb]{0.00,0.00,0.50}{##1}}}
\expandafter\def\csname PY@tok@go\endcsname{\def\PY@tc##1{\textcolor[rgb]{0.53,0.53,0.53}{##1}}}
\expandafter\def\csname PY@tok@gt\endcsname{\def\PY@tc##1{\textcolor[rgb]{0.00,0.27,0.87}{##1}}}
\expandafter\def\csname PY@tok@err\endcsname{\def\PY@bc##1{\setlength{\fboxsep}{0pt}\fcolorbox[rgb]{1.00,0.00,0.00}{1,1,1}{\strut ##1}}}
\expandafter\def\csname PY@tok@kc\endcsname{\let\PY@bf=\textbf\def\PY@tc##1{\textcolor[rgb]{0.00,0.50,0.00}{##1}}}
\expandafter\def\csname PY@tok@kd\endcsname{\let\PY@bf=\textbf\def\PY@tc##1{\textcolor[rgb]{0.00,0.50,0.00}{##1}}}
\expandafter\def\csname PY@tok@kn\endcsname{\let\PY@bf=\textbf\def\PY@tc##1{\textcolor[rgb]{0.00,0.50,0.00}{##1}}}
\expandafter\def\csname PY@tok@kr\endcsname{\let\PY@bf=\textbf\def\PY@tc##1{\textcolor[rgb]{0.00,0.50,0.00}{##1}}}
\expandafter\def\csname PY@tok@bp\endcsname{\def\PY@tc##1{\textcolor[rgb]{0.00,0.50,0.00}{##1}}}
\expandafter\def\csname PY@tok@fm\endcsname{\def\PY@tc##1{\textcolor[rgb]{0.00,0.00,1.00}{##1}}}
\expandafter\def\csname PY@tok@vc\endcsname{\def\PY@tc##1{\textcolor[rgb]{0.10,0.09,0.49}{##1}}}
\expandafter\def\csname PY@tok@vg\endcsname{\def\PY@tc##1{\textcolor[rgb]{0.10,0.09,0.49}{##1}}}
\expandafter\def\csname PY@tok@vi\endcsname{\def\PY@tc##1{\textcolor[rgb]{0.10,0.09,0.49}{##1}}}
\expandafter\def\csname PY@tok@vm\endcsname{\def\PY@tc##1{\textcolor[rgb]{0.10,0.09,0.49}{##1}}}
\expandafter\def\csname PY@tok@sa\endcsname{\def\PY@tc##1{\textcolor[rgb]{0.73,0.13,0.13}{##1}}}
\expandafter\def\csname PY@tok@sb\endcsname{\def\PY@tc##1{\textcolor[rgb]{0.73,0.13,0.13}{##1}}}
\expandafter\def\csname PY@tok@sc\endcsname{\def\PY@tc##1{\textcolor[rgb]{0.73,0.13,0.13}{##1}}}
\expandafter\def\csname PY@tok@dl\endcsname{\def\PY@tc##1{\textcolor[rgb]{0.73,0.13,0.13}{##1}}}
\expandafter\def\csname PY@tok@s2\endcsname{\def\PY@tc##1{\textcolor[rgb]{0.73,0.13,0.13}{##1}}}
\expandafter\def\csname PY@tok@sh\endcsname{\def\PY@tc##1{\textcolor[rgb]{0.73,0.13,0.13}{##1}}}
\expandafter\def\csname PY@tok@s1\endcsname{\def\PY@tc##1{\textcolor[rgb]{0.73,0.13,0.13}{##1}}}
\expandafter\def\csname PY@tok@mb\endcsname{\def\PY@tc##1{\textcolor[rgb]{0.40,0.40,0.40}{##1}}}
\expandafter\def\csname PY@tok@mf\endcsname{\def\PY@tc##1{\textcolor[rgb]{0.40,0.40,0.40}{##1}}}
\expandafter\def\csname PY@tok@mh\endcsname{\def\PY@tc##1{\textcolor[rgb]{0.40,0.40,0.40}{##1}}}
\expandafter\def\csname PY@tok@mi\endcsname{\def\PY@tc##1{\textcolor[rgb]{0.40,0.40,0.40}{##1}}}
\expandafter\def\csname PY@tok@il\endcsname{\def\PY@tc##1{\textcolor[rgb]{0.40,0.40,0.40}{##1}}}
\expandafter\def\csname PY@tok@mo\endcsname{\def\PY@tc##1{\textcolor[rgb]{0.40,0.40,0.40}{##1}}}
\expandafter\def\csname PY@tok@ch\endcsname{\let\PY@it=\textit\def\PY@tc##1{\textcolor[rgb]{0.25,0.50,0.50}{##1}}}
\expandafter\def\csname PY@tok@cm\endcsname{\let\PY@it=\textit\def\PY@tc##1{\textcolor[rgb]{0.25,0.50,0.50}{##1}}}
\expandafter\def\csname PY@tok@cpf\endcsname{\let\PY@it=\textit\def\PY@tc##1{\textcolor[rgb]{0.25,0.50,0.50}{##1}}}
\expandafter\def\csname PY@tok@c1\endcsname{\let\PY@it=\textit\def\PY@tc##1{\textcolor[rgb]{0.25,0.50,0.50}{##1}}}
\expandafter\def\csname PY@tok@cs\endcsname{\let\PY@it=\textit\def\PY@tc##1{\textcolor[rgb]{0.25,0.50,0.50}{##1}}}

\def\PYZbs{\char`\\}
\def\PYZus{\char`\_}
\def\PYZob{\char`\{}
\def\PYZcb{\char`\}}
\def\PYZca{\char`\^}
\def\PYZam{\char`\&}
\def\PYZlt{\char`\<}
\def\PYZgt{\char`\>}
\def\PYZsh{\char`\#}
\def\PYZpc{\char`\%}
\def\PYZdl{\char`\$}
\def\PYZhy{\char`\-}
\def\PYZsq{\char`\'}
\def\PYZdq{\char`\"}
\def\PYZti{\char`\~}
% for compatibility with earlier versions
\def\PYZat{@}
\def\PYZlb{[}
\def\PYZrb{]}
\makeatother


    % Exact colors from NB
    \definecolor{incolor}{rgb}{0.0, 0.0, 0.5}
    \definecolor{outcolor}{rgb}{0.545, 0.0, 0.0}



    
    % Prevent overflowing lines due to hard-to-break entities
    \sloppy 
    % Setup hyperref package
    \hypersetup{
      breaklinks=true,  % so long urls are correctly broken across lines
      colorlinks=true,
      urlcolor=urlcolor,
      linkcolor=linkcolor,
      citecolor=citecolor,
      }
    % Slightly bigger margins than the latex defaults
    
    \geometry{verbose,tmargin=1in,bmargin=1in,lmargin=1in,rmargin=1in}
    
    

    \begin{document}
    
	\title{数值分析实验三}
	\author{计63\,\,陈晟祺\,\,2016010981}
    
	
    \maketitle
    
    

    

    \subsection{上机题6}\label{ux4e0aux673aux98986}

\subsubsection{实验概要}\label{ux5b9eux9a8cux6982ux8981}

本题要求使用 Cholesky 分解方法求解方程,并计算残差的误差的
\(\infty\)-范数,并在施加扰动和矩阵维度变化的情况下重复这一过程。

    \subsection{实验过程}\label{ux5b9eux9a8cux8fc7ux7a0b}

首先导入常用库

    \begin{Verbatim}[commandchars=\\\{\}]
{\color{incolor}In [{\color{incolor}1}]:} \PY{k+kn}{import} \PY{n+nn}{numpy} \PY{k}{as} \PY{n+nn}{np}
\end{Verbatim}


    生成 Hilbert 矩阵 \(\mathbf{H}_n\),并计算
\(\mathbf{b}=\mathbf{H}_n\mathbf{1}\):

    \begin{Verbatim}[commandchars=\\\{\}]
{\color{incolor}In [{\color{incolor}2}]:} \PY{n}{n} \PY{o}{=} \PY{l+m+mi}{10}
        \PY{n}{H} \PY{o}{=} \PY{n}{np}\PY{o}{.}\PY{n}{fromfunction}\PY{p}{(}\PY{k}{lambda} \PY{n}{i}\PY{p}{,} \PY{n}{j}\PY{p}{:} \PY{l+m+mf}{1.} \PY{o}{/} \PY{p}{(}\PY{n}{i} \PY{o}{+} \PY{n}{j} \PY{o}{+} \PY{l+m+mi}{1}\PY{p}{)}\PY{p}{,} \PY{p}{(}\PY{n}{n}\PY{p}{,}\PY{n}{n}\PY{p}{)}\PY{p}{)}
        \PY{n}{ones} \PY{o}{=} \PY{n}{np}\PY{o}{.}\PY{n}{ones}\PY{p}{(}\PY{n}{n}\PY{p}{)}
        \PY{n}{b} \PY{o}{=} \PY{n}{np}\PY{o}{.}\PY{n}{dot}\PY{p}{(}\PY{n}{H}\PY{p}{,} \PY{n}{ones}\PY{p}{)}
\end{Verbatim}


    按照算法 3.10 的描述,对 H 进行 Cholesky 分解:

    \begin{Verbatim}[commandchars=\\\{\}]
{\color{incolor}In [{\color{incolor}3}]:} \PY{k}{def} \PY{n+nf}{chole}\PY{p}{(}\PY{n}{M}\PY{p}{)}\PY{p}{:}
            \PY{n}{n} \PY{o}{=} \PY{n}{np}\PY{o}{.}\PY{n}{shape}\PY{p}{(}\PY{n}{M}\PY{p}{)}\PY{p}{[}\PY{l+m+mi}{0}\PY{p}{]}
            \PY{n}{L} \PY{o}{=} \PY{n}{np}\PY{o}{.}\PY{n}{zeros\PYZus{}like}\PY{p}{(}\PY{n}{M}\PY{p}{)} \PY{c+c1}{\PYZsh{} avoid damaging H}
            
            \PY{k}{for} \PY{n}{j} \PY{o+ow}{in} \PY{n+nb}{range}\PY{p}{(}\PY{n}{n}\PY{p}{)}\PY{p}{:}
                \PY{n}{L}\PY{p}{[}\PY{n}{j}\PY{p}{]}\PY{p}{[}\PY{n}{j}\PY{p}{]} \PY{o}{=} \PY{n}{M}\PY{p}{[}\PY{n}{j}\PY{p}{]}\PY{p}{[}\PY{n}{j}\PY{p}{]}
                \PY{k}{for} \PY{n}{k} \PY{o+ow}{in} \PY{n+nb}{range}\PY{p}{(}\PY{l+m+mi}{0}\PY{p}{,} \PY{n}{j}\PY{p}{)}\PY{p}{:}
                    \PY{n}{L}\PY{p}{[}\PY{n}{j}\PY{p}{]}\PY{p}{[}\PY{n}{j}\PY{p}{]} \PY{o}{\PYZhy{}}\PY{o}{=} \PY{n}{L}\PY{p}{[}\PY{n}{j}\PY{p}{]}\PY{p}{[}\PY{n}{k}\PY{p}{]} \PY{o}{*}\PY{o}{*} \PY{l+m+mi}{2}
                \PY{n}{L}\PY{p}{[}\PY{n}{j}\PY{p}{]}\PY{p}{[}\PY{n}{j}\PY{p}{]} \PY{o}{=} \PY{n}{np}\PY{o}{.}\PY{n}{sqrt}\PY{p}{(}\PY{n}{L}\PY{p}{[}\PY{n}{j}\PY{p}{]}\PY{p}{[}\PY{n}{j}\PY{p}{]}\PY{p}{)}
                \PY{k}{for} \PY{n}{i} \PY{o+ow}{in} \PY{n+nb}{range}\PY{p}{(}\PY{n}{j} \PY{o}{+} \PY{l+m+mi}{1}\PY{p}{,} \PY{n}{n}\PY{p}{)}\PY{p}{:}
                    \PY{n}{L}\PY{p}{[}\PY{n}{i}\PY{p}{]}\PY{p}{[}\PY{n}{j}\PY{p}{]} \PY{o}{=} \PY{n}{M}\PY{p}{[}\PY{n}{i}\PY{p}{]}\PY{p}{[}\PY{n}{j}\PY{p}{]}
                    \PY{k}{for} \PY{n}{k} \PY{o+ow}{in} \PY{n+nb}{range}\PY{p}{(}\PY{l+m+mi}{0}\PY{p}{,} \PY{n}{j}\PY{p}{)}\PY{p}{:}
                        \PY{n}{L}\PY{p}{[}\PY{n}{i}\PY{p}{]}\PY{p}{[}\PY{n}{j}\PY{p}{]} \PY{o}{\PYZhy{}}\PY{o}{=} \PY{n}{L}\PY{p}{[}\PY{n}{i}\PY{p}{]}\PY{p}{[}\PY{n}{k}\PY{p}{]} \PY{o}{*} \PY{n}{L}\PY{p}{[}\PY{n}{j}\PY{p}{]}\PY{p}{[}\PY{n}{k}\PY{p}{]}
                    \PY{n}{L}\PY{p}{[}\PY{n}{i}\PY{p}{]}\PY{p}{[}\PY{n}{j}\PY{p}{]} \PY{o}{/}\PY{o}{=} \PY{n}{L}\PY{p}{[}\PY{n}{j}\PY{p}{]}\PY{p}{[}\PY{n}{j}\PY{p}{]}
                    
            \PY{k}{return} \PY{n}{L}
\end{Verbatim}


    \begin{Verbatim}[commandchars=\\\{\}]
{\color{incolor}In [{\color{incolor}4}]:} \PY{n}{L} \PY{o}{=} \PY{n}{chole}\PY{p}{(}\PY{n}{H}\PY{p}{)}
\end{Verbatim}


    此时 \(\mathbf{H_n} = \mathbf{LL^\intercal}\),即
\(\mathbf{LL^\intercal} \mathbf{x}=\mathbf{b}\)。因此先按照算法3.7求解
\(\mathbf{Ly=b}\)(注意此时对角线元素并非都为1),再使用算法3.2求解
\(\mathbf{L^\intercal x=y}\) 即可:

    \begin{Verbatim}[commandchars=\\\{\}]
{\color{incolor}In [{\color{incolor}5}]:} \PY{k}{def} \PY{n+nf}{solve\PYZus{}L}\PY{p}{(}\PY{n}{L}\PY{p}{,} \PY{n}{b}\PY{p}{)}\PY{p}{:}
            \PY{n}{n} \PY{o}{=} \PY{n}{np}\PY{o}{.}\PY{n}{shape}\PY{p}{(}\PY{n}{b}\PY{p}{)}\PY{p}{[}\PY{l+m+mi}{0}\PY{p}{]}
            \PY{n}{y} \PY{o}{=} \PY{n}{np}\PY{o}{.}\PY{n}{zeros\PYZus{}like}\PY{p}{(}\PY{n}{b}\PY{p}{)}
            \PY{k}{for} \PY{n}{i} \PY{o+ow}{in} \PY{n+nb}{range}\PY{p}{(}\PY{n}{n}\PY{p}{)}\PY{p}{:}
                \PY{n}{y}\PY{p}{[}\PY{n}{i}\PY{p}{]} \PY{o}{=} \PY{n}{b}\PY{p}{[}\PY{n}{i}\PY{p}{]}
                \PY{k}{for} \PY{n}{j} \PY{o+ow}{in} \PY{n+nb}{range}\PY{p}{(}\PY{l+m+mi}{0}\PY{p}{,} \PY{n}{i}\PY{p}{)}\PY{p}{:}
                    \PY{n}{y}\PY{p}{[}\PY{n}{i}\PY{p}{]} \PY{o}{\PYZhy{}}\PY{o}{=} \PY{n}{L}\PY{p}{[}\PY{n}{i}\PY{p}{]}\PY{p}{[}\PY{n}{j}\PY{p}{]} \PY{o}{*} \PY{n}{y}\PY{p}{[}\PY{n}{j}\PY{p}{]}
                \PY{n}{y}\PY{p}{[}\PY{n}{i}\PY{p}{]} \PY{o}{/}\PY{o}{=} \PY{n}{L}\PY{p}{[}\PY{n}{i}\PY{p}{]}\PY{p}{[}\PY{n}{i}\PY{p}{]}
        
            \PY{n}{x} \PY{o}{=} \PY{n}{np}\PY{o}{.}\PY{n}{zeros\PYZus{}like}\PY{p}{(}\PY{n}{b}\PY{p}{)}
            \PY{k}{for} \PY{n}{i} \PY{o+ow}{in} \PY{n+nb}{reversed}\PY{p}{(}\PY{n+nb}{range}\PY{p}{(}\PY{n}{n}\PY{p}{)}\PY{p}{)}\PY{p}{:}
                \PY{n}{x}\PY{p}{[}\PY{n}{i}\PY{p}{]} \PY{o}{=} \PY{n}{y}\PY{p}{[}\PY{n}{i}\PY{p}{]}
                \PY{k}{for} \PY{n}{j} \PY{o+ow}{in} \PY{n+nb}{reversed}\PY{p}{(}\PY{n+nb}{range}\PY{p}{(}\PY{n}{i} \PY{o}{+} \PY{l+m+mi}{1}\PY{p}{,} \PY{n}{n}\PY{p}{)}\PY{p}{)}\PY{p}{:}
                    \PY{n}{x}\PY{p}{[}\PY{n}{i}\PY{p}{]} \PY{o}{\PYZhy{}}\PY{o}{=} \PY{n}{L}\PY{p}{[}\PY{n}{j}\PY{p}{]}\PY{p}{[}\PY{n}{i}\PY{p}{]} \PY{o}{*} \PY{n}{x}\PY{p}{[}\PY{n}{j}\PY{p}{]} \PY{c+c1}{\PYZsh{} actually use L\PYZca{}T}
                \PY{n}{x}\PY{p}{[}\PY{n}{i}\PY{p}{]} \PY{o}{/}\PY{o}{=} \PY{n}{L}\PY{p}{[}\PY{n}{i}\PY{p}{]}\PY{p}{[}\PY{n}{i}\PY{p}{]}
            
            \PY{k}{return} \PY{n}{x}
\end{Verbatim}


    下面计算残差和误差:

    \begin{Verbatim}[commandchars=\\\{\}]
{\color{incolor}In [{\color{incolor}6}]:} \PY{n}{x} \PY{o}{=} \PY{n}{solve\PYZus{}L}\PY{p}{(}\PY{n}{L}\PY{p}{,} \PY{n}{b}\PY{p}{)}
        \PY{n}{r} \PY{o}{=} \PY{n}{np}\PY{o}{.}\PY{n}{max}\PY{p}{(}\PY{n}{np}\PY{o}{.}\PY{n}{abs}\PY{p}{(}\PY{n}{b} \PY{o}{\PYZhy{}} \PY{n}{np}\PY{o}{.}\PY{n}{dot}\PY{p}{(}\PY{n}{H}\PY{p}{,} \PY{n}{x}\PY{p}{)}\PY{p}{)}\PY{p}{)}
        \PY{n}{delta} \PY{o}{=} \PY{n}{np}\PY{o}{.}\PY{n}{max}\PY{p}{(}\PY{n}{np}\PY{o}{.}\PY{n}{abs}\PY{p}{(}\PY{n}{ones} \PY{o}{\PYZhy{}} \PY{n}{x}\PY{p}{)}\PY{p}{)}
        \PY{n+nb}{print}\PY{p}{(}\PY{l+s+s2}{\PYZdq{}}\PY{l+s+s2}{r = }\PY{l+s+si}{\PYZob{}:.20f\PYZcb{}}\PY{l+s+s2}{, delta = }\PY{l+s+si}{\PYZob{}:.20f\PYZcb{}}\PY{l+s+s2}{\PYZdq{}}\PY{o}{.}\PY{n}{format}\PY{p}{(}\PY{n}{r}\PY{p}{,} \PY{n}{delta}\PY{p}{)}\PY{p}{)}
\end{Verbatim}


    \begin{Verbatim}[commandchars=\\\{\}]
r = 0.00000000000000022204, delta = 0.00044458507134448322

    \end{Verbatim}

    当右端项有扰动(正态噪音)时,重复上述过程:

    \begin{Verbatim}[commandchars=\\\{\}]
{\color{incolor}In [{\color{incolor}7}]:} \PY{n}{x} \PY{o}{=} \PY{n}{solve\PYZus{}L}\PY{p}{(}\PY{n}{L}\PY{p}{,} \PY{n}{b} \PY{o}{+} \PY{n}{np}\PY{o}{.}\PY{n}{random}\PY{o}{.}\PY{n}{normal}\PY{p}{(}\PY{l+m+mi}{0}\PY{p}{,} \PY{l+m+mf}{1e\PYZhy{}7}\PY{p}{,} \PY{n}{n}\PY{p}{)}\PY{p}{)}
        \PY{n}{r} \PY{o}{=} \PY{n}{np}\PY{o}{.}\PY{n}{max}\PY{p}{(}\PY{n}{np}\PY{o}{.}\PY{n}{abs}\PY{p}{(}\PY{n}{b} \PY{o}{\PYZhy{}} \PY{n}{np}\PY{o}{.}\PY{n}{dot}\PY{p}{(}\PY{n}{H}\PY{p}{,} \PY{n}{x}\PY{p}{)}\PY{p}{)}\PY{p}{)}
        \PY{n}{delta} \PY{o}{=} \PY{n}{np}\PY{o}{.}\PY{n}{max}\PY{p}{(}\PY{n}{np}\PY{o}{.}\PY{n}{abs}\PY{p}{(}\PY{n}{ones} \PY{o}{\PYZhy{}} \PY{n}{x}\PY{p}{)}\PY{p}{)}
        \PY{n+nb}{print}\PY{p}{(}\PY{l+s+s2}{\PYZdq{}}\PY{l+s+s2}{r = }\PY{l+s+si}{\PYZob{}:.20f\PYZcb{}}\PY{l+s+s2}{, delta = }\PY{l+s+si}{\PYZob{}:.20f\PYZcb{}}\PY{l+s+s2}{\PYZdq{}}\PY{o}{.}\PY{n}{format}\PY{p}{(}\PY{n}{r}\PY{p}{,} \PY{n}{delta}\PY{p}{)}\PY{p}{)}
\end{Verbatim}


    \begin{Verbatim}[commandchars=\\\{\}]
r = 0.00000017734695356708, delta = 54922.46394105027138721198

    \end{Verbatim}

    可见如果 \(b\)
发生扰动,将产生极大的误差,而残差依旧很小。也就是说,在扰动意义下的解依旧是正确的,但是与原本的解差别很大。这说明关于
\(H_n\)
矩阵的方程问题敏感性很大,这种矩阵是病态的。可以通过计算矩阵的条件数来观察到这一结论:

    \begin{Verbatim}[commandchars=\\\{\}]
{\color{incolor}In [{\color{incolor}8}]:} \PY{n}{np}\PY{o}{.}\PY{n}{linalg}\PY{o}{.}\PY{n}{cond}\PY{p}{(}\PY{n}{H}\PY{p}{,} \PY{n}{p}\PY{o}{=}\PY{n}{np}\PY{o}{.}\PY{n}{inf}\PY{p}{)}
\end{Verbatim}


\begin{Verbatim}[commandchars=\\\{\}]
{\color{outcolor}Out[{\color{outcolor}8}]:} 35353724553756.422
\end{Verbatim}
            
    下面对于不同的 n 计算得到的解的残差和误差:

    \begin{Verbatim}[commandchars=\\\{\}]
{\color{incolor}In [{\color{incolor}9}]:} \PY{k}{def} \PY{n+nf}{chole\PYZus{}solve}\PY{p}{(}\PY{n}{n}\PY{p}{)}\PY{p}{:}
            \PY{n}{H} \PY{o}{=} \PY{n}{np}\PY{o}{.}\PY{n}{fromfunction}\PY{p}{(}\PY{k}{lambda} \PY{n}{i}\PY{p}{,} \PY{n}{j}\PY{p}{:} \PY{l+m+mf}{1.} \PY{o}{/} \PY{p}{(}\PY{n}{i} \PY{o}{+} \PY{n}{j} \PY{o}{+} \PY{l+m+mi}{1}\PY{p}{)}\PY{p}{,} \PY{p}{(}\PY{n}{n}\PY{p}{,}\PY{n}{n}\PY{p}{)}\PY{p}{)}
            \PY{n}{cond} \PY{o}{=} \PY{n}{np}\PY{o}{.}\PY{n}{linalg}\PY{o}{.}\PY{n}{cond}\PY{p}{(}\PY{n}{H}\PY{p}{,} \PY{n}{p}\PY{o}{=}\PY{n}{np}\PY{o}{.}\PY{n}{inf}\PY{p}{)}
            \PY{n+nb}{print}\PY{p}{(}\PY{l+s+s1}{\PYZsq{}}\PY{l+s+s1}{n = }\PY{l+s+si}{\PYZob{}\PYZcb{}}\PY{l+s+s1}{,}\PY{l+s+se}{\PYZbs{}t}\PY{l+s+s1}{cond = }\PY{l+s+si}{\PYZob{}\PYZcb{}}\PY{l+s+s1}{\PYZsq{}}\PY{o}{.}\PY{n}{format}\PY{p}{(}\PY{n}{n}\PY{p}{,} \PY{n}{cond}\PY{p}{)}\PY{p}{)}
            \PY{n}{ones} \PY{o}{=} \PY{n}{np}\PY{o}{.}\PY{n}{ones}\PY{p}{(}\PY{n}{n}\PY{p}{)}
            \PY{n}{b} \PY{o}{=} \PY{n}{np}\PY{o}{.}\PY{n}{dot}\PY{p}{(}\PY{n}{H}\PY{p}{,} \PY{n}{ones}\PY{p}{)}
            \PY{n}{L} \PY{o}{=} \PY{n}{chole}\PY{p}{(}\PY{n}{H}\PY{p}{)}
            \PY{n}{x} \PY{o}{=} \PY{n}{solve\PYZus{}L}\PY{p}{(}\PY{n}{L}\PY{p}{,} \PY{n}{b}\PY{p}{)}
            \PY{n}{r} \PY{o}{=} \PY{n}{np}\PY{o}{.}\PY{n}{max}\PY{p}{(}\PY{n}{np}\PY{o}{.}\PY{n}{abs}\PY{p}{(}\PY{n}{b} \PY{o}{\PYZhy{}} \PY{n}{np}\PY{o}{.}\PY{n}{dot}\PY{p}{(}\PY{n}{H}\PY{p}{,} \PY{n}{x}\PY{p}{)}\PY{p}{)}\PY{p}{)}
            \PY{n}{delta} \PY{o}{=} \PY{n}{np}\PY{o}{.}\PY{n}{max}\PY{p}{(}\PY{n}{np}\PY{o}{.}\PY{n}{abs}\PY{p}{(}\PY{n}{ones} \PY{o}{\PYZhy{}} \PY{n}{x}\PY{p}{)}\PY{p}{)}
            \PY{n+nb}{print}\PY{p}{(}\PY{l+s+s2}{\PYZdq{}}\PY{l+s+s2}{Original:}\PY{l+s+se}{\PYZbs{}t}\PY{l+s+s2}{r = }\PY{l+s+si}{\PYZob{}:.20f\PYZcb{}}\PY{l+s+s2}{, delta = }\PY{l+s+si}{\PYZob{}:.20f\PYZcb{}}\PY{l+s+s2}{\PYZdq{}}\PY{o}{.}\PY{n}{format}\PY{p}{(}\PY{n}{r}\PY{p}{,} \PY{n}{delta}\PY{p}{)}\PY{p}{)}
            \PY{n}{x\PYZus{}dist} \PY{o}{=} \PY{n}{solve\PYZus{}L}\PY{p}{(}\PY{n}{L}\PY{p}{,} \PY{n}{b} \PY{o}{+} \PY{n}{np}\PY{o}{.}\PY{n}{random}\PY{o}{.}\PY{n}{normal}\PY{p}{(}\PY{l+m+mi}{0}\PY{p}{,} \PY{l+m+mf}{1e\PYZhy{}7}\PY{p}{,} \PY{n}{n}\PY{p}{)}\PY{p}{)}
            \PY{n}{r\PYZus{}dist} \PY{o}{=} \PY{n}{np}\PY{o}{.}\PY{n}{max}\PY{p}{(}\PY{n}{np}\PY{o}{.}\PY{n}{abs}\PY{p}{(}\PY{n}{b} \PY{o}{\PYZhy{}} \PY{n}{np}\PY{o}{.}\PY{n}{dot}\PY{p}{(}\PY{n}{H}\PY{p}{,} \PY{n}{x\PYZus{}dist}\PY{p}{)}\PY{p}{)}\PY{p}{)}
            \PY{n}{delta\PYZus{}dist} \PY{o}{=} \PY{n}{np}\PY{o}{.}\PY{n}{max}\PY{p}{(}\PY{n}{np}\PY{o}{.}\PY{n}{abs}\PY{p}{(}\PY{n}{ones} \PY{o}{\PYZhy{}} \PY{n}{x\PYZus{}dist}\PY{p}{)}\PY{p}{)}
            \PY{n+nb}{print}\PY{p}{(}\PY{l+s+s2}{\PYZdq{}}\PY{l+s+s2}{Disturbed:}\PY{l+s+se}{\PYZbs{}t}\PY{l+s+s2}{r = }\PY{l+s+si}{\PYZob{}:.20f\PYZcb{}}\PY{l+s+s2}{, delta = }\PY{l+s+si}{\PYZob{}:.20f\PYZcb{}}\PY{l+s+s2}{\PYZdq{}}\PY{o}{.}\PY{n}{format}\PY{p}{(}\PY{n}{r\PYZus{}dist}\PY{p}{,} \PY{n}{delta\PYZus{}dist}\PY{p}{)}\PY{p}{)}
            
            \PY{k}{def} \PY{n+nf}{f\PYZus{}to\PYZus{}str}\PY{p}{(}\PY{n}{f}\PY{p}{)}\PY{p}{:}
                \PY{k}{return} \PY{l+s+s1}{\PYZsq{}}\PY{l+s+si}{\PYZob{}:.4f\PYZcb{}}\PY{l+s+s1}{\PYZsq{}}\PY{o}{.}\PY{n}{format}\PY{p}{(}\PY{n}{f}\PY{p}{)}
            
            \PY{k}{with} \PY{n+nb}{open}\PY{p}{(}\PY{l+s+s1}{\PYZsq{}}\PY{l+s+s1}{result\PYZus{}n=}\PY{l+s+si}{\PYZob{}\PYZcb{}}\PY{l+s+s1}{.txt}\PY{l+s+s1}{\PYZsq{}}\PY{o}{.}\PY{n}{format}\PY{p}{(}\PY{n}{n}\PY{p}{)}\PY{p}{,} \PY{l+s+s1}{\PYZsq{}}\PY{l+s+s1}{w}\PY{l+s+s1}{\PYZsq{}}\PY{p}{)} \PY{k}{as} \PY{n}{f}\PY{p}{:}
                \PY{n}{f}\PY{o}{.}\PY{n}{write}\PY{p}{(}\PY{l+s+s1}{\PYZsq{}}\PY{l+s+s1}{Original:}\PY{l+s+se}{\PYZbs{}t}\PY{l+s+s1}{\PYZsq{}}\PY{p}{)}
                \PY{n}{f}\PY{o}{.}\PY{n}{write}\PY{p}{(}\PY{l+s+s1}{\PYZsq{}}\PY{l+s+se}{\PYZbs{}t}\PY{l+s+s1}{\PYZsq{}}\PY{o}{.}\PY{n}{join}\PY{p}{(}\PY{n+nb}{map}\PY{p}{(}\PY{n}{f\PYZus{}to\PYZus{}str}\PY{p}{,} \PY{n}{x}\PY{p}{)}\PY{p}{)} \PY{o}{+} \PY{l+s+s1}{\PYZsq{}}\PY{l+s+se}{\PYZbs{}n}\PY{l+s+s1}{\PYZsq{}}\PY{p}{)}
                \PY{n}{f}\PY{o}{.}\PY{n}{write}\PY{p}{(}\PY{l+s+s1}{\PYZsq{}}\PY{l+s+s1}{Disturbed:}\PY{l+s+se}{\PYZbs{}t}\PY{l+s+s1}{\PYZsq{}}\PY{p}{)}
                \PY{n}{f}\PY{o}{.}\PY{n}{write}\PY{p}{(}\PY{l+s+s1}{\PYZsq{}}\PY{l+s+se}{\PYZbs{}t}\PY{l+s+s1}{\PYZsq{}}\PY{o}{.}\PY{n}{join}\PY{p}{(}\PY{n+nb}{map}\PY{p}{(}\PY{n}{f\PYZus{}to\PYZus{}str}\PY{p}{,} \PY{n}{x\PYZus{}dist}\PY{p}{)}\PY{p}{)} \PY{o}{+} \PY{l+s+s1}{\PYZsq{}}\PY{l+s+se}{\PYZbs{}n}\PY{l+s+s1}{\PYZsq{}}\PY{p}{)}
\end{Verbatim}


    \begin{Verbatim}[commandchars=\\\{\}]
{\color{incolor}In [{\color{incolor}10}]:} \PY{n}{chole\PYZus{}solve}\PY{p}{(}\PY{l+m+mi}{8}\PY{p}{)}
         \PY{n}{chole\PYZus{}solve}\PY{p}{(}\PY{l+m+mi}{10}\PY{p}{)}
         \PY{n}{chole\PYZus{}solve}\PY{p}{(}\PY{l+m+mi}{12}\PY{p}{)}
\end{Verbatim}


    \begin{Verbatim}[commandchars=\\\{\}]
n = 8,	cond = 33872790819.49471
Original:	r = 0.00000000000000022204, delta = 0.00000041154382102171
Disturbed:	r = 0.00000020867323935470, delta = 126.71291104727328047375
n = 10,	cond = 35353724553756.42
Original:	r = 0.00000000000000022204, delta = 0.00044458507134448322
Disturbed:	r = 0.00000015520206209096, delta = 381008.03898283827584236860
n = 12,	cond = 3.798320122691213e+16
Original:	r = 0.00000000000000044409, delta = 0.33580581043297352828
Disturbed:	r = 0.00000025662089786493, delta = 14155193.01611004024744033813

    \end{Verbatim}

    由上面的结果可知,当 \(n\) 越大,\(H_n\)
的条件数越大,并且解的残差、误差都越大,并且施以同样的扰动时,带来的误差也越大。这与
3.4 给出的结论是相符的。得到的详细解可见目录下的各 txt 文件。

    \subsubsection{实验结论}\label{ux5b9eux9a8cux7ed3ux8bba}

通过本次实验,我实现了正定矩阵的 Cholesky
分解并使用结果进行方程求解。同时,通过对带扰动的 Hilbert
矩阵的方程求解,我们能体会到它的强敏感性;并且随矩阵维度增加,病态性会变得更强。因此,受限制于浮点运算的误差,此类矩阵方程问题事实上很难得到可接受的解。

本次实验中,由于原有矩阵尚有用处,因此我实现的 Cholesky
分解并非原地的。在矩阵规模较大时,应当原地存储系数,或者使用稀疏矩阵,从而节省空间。


    % Add a bibliography block to the postdoc
    
    
    
    \end{document}
